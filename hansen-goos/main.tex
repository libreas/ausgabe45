\documentclass[a4paper,
fontsize=11pt,
%headings=small,
oneside,
numbers=noperiodatend,
parskip=half-,
bibliography=totoc,
final
]{scrartcl}

\usepackage[babel]{csquotes}
\usepackage{synttree}
\usepackage{graphicx}
\setkeys{Gin}{width=.4\textwidth} %default pics size

\graphicspath{{./plots/}}
\usepackage[ngerman]{babel}
\usepackage[T1]{fontenc}
%\usepackage{amsmath}
\usepackage[utf8x]{inputenc}
\usepackage [hyphens]{url}
\usepackage{booktabs} 
\usepackage[left=2.4cm,right=2.4cm,top=2.3cm,bottom=2cm,includeheadfoot]{geometry}
\usepackage[labelformat=empty]{caption} % option 'labelformat=empty]' to surpress adding "Abbildung 1:" or "Figure 1" before each caption / use parameter '\captionsetup{labelformat=empty}' instead to change this for just one caption
\usepackage{eurosym}
\usepackage{multirow}
\usepackage[ngerman]{varioref}
\setcapindent{1em}
\renewcommand{\labelitemi}{--}
\usepackage{paralist}
\usepackage{pdfpages}
\usepackage{lscape}
\usepackage{float}
\usepackage{acronym}
\usepackage{eurosym}
\usepackage{longtable,lscape}
\usepackage{mathpazo}
\usepackage[normalem]{ulem} %emphasize weiterhin kursiv
\usepackage[flushmargin,ragged]{footmisc} % left align footnote
\usepackage{ccicons} 
\setcapindent{0pt} % no indentation in captions
\usepackage{xurl} % Breaks URLs

%%%% fancy LIBREAS URL color 
\usepackage{xcolor}
\definecolor{libreas}{RGB}{112,0,0}

\usepackage{listings}

\urlstyle{same}  % don't use monospace font for urls

\usepackage[fleqn]{amsmath}

%adjust fontsize for part

\usepackage{sectsty}
\partfont{\large}

%Das BibTeX-Zeichen mit \BibTeX setzen:
\def\symbol#1{\char #1\relax}
\def\bsl{{\tt\symbol{'134}}}
\def\BibTeX{{\rm B\kern-.05em{\sc i\kern-.025em b}\kern-.08em
    T\kern-.1667em\lower.7ex\hbox{E}\kern-.125emX}}

\usepackage{fancyhdr}
\fancyhf{}
\pagestyle{fancyplain}
\fancyhead[R]{\thepage}

% make sure bookmarks are created eventough sections are not numbered!
% uncommend if sections are numbered (bookmarks created by default)
\makeatletter
\renewcommand\@seccntformat[1]{}
\makeatother

% typo setup
\clubpenalty = 10000
\widowpenalty = 10000
\displaywidowpenalty = 10000

\usepackage{hyperxmp}
\usepackage[colorlinks, linkcolor=black,citecolor=black, urlcolor=libreas,
breaklinks= true,bookmarks=true,bookmarksopen=true]{hyperref}
\usepackage{breakurl}

%meta
%meta

\fancyhead[L]{I. Hansen-Goos et al.\\ %author
LIBREAS. Library Ideas, 45 (2024). % journal, issue, volume.
\href{https://doi.org/10.18452/...}{\color{black}https://doi.org/10.18452/...}
{}} % doi 
\fancyhead[R]{\thepage} %page number
\fancyfoot[L] {\ccLogo \ccAttribution\ \href{https://creativecommons.org/licenses/by/4.0/}{\color{black}Creative Commons BY 4.0}}  %licence
\fancyfoot[R] {ISSN: 1860-7950}

\title{\LARGE{Hinter den Geräuschkulissen –- ein Projekt zur akustischen Vermessung des Scharoun-Gebäudes der Staatsbibliothek zu Berlin}} % author
\author{Ina Hansen-Goos, Barbara Heindl, Harald Krewer, Christian Mathieu}

\setcounter{page}{1}

\hypersetup{%
      pdftitle={Hinter den Geräuschkulissen – ein Projekt zur akustischen Vermessung 
des Scharoun-Gebäudes der Staatsbibliothek zu Berlin},
      pdfcopyright={CC BY 4.0 International},
      pdfsubject={LIBREAS. Library Ideas, 45 (2024).},
      pdfkeywords={Stabi Berlin, Soundscape, Staatsbibliothek Preussischer Kulturbesitz},
      pdflicenseurl={https://creativecommons.org/licenses/by/4.0/},
      pdfurl={https://doi.org/10.18452/...},
      pdfdoi={10.18452/...},
      pdflang={de},
      pdfmetalang={de}
     }



\date{}
\begin{document}

\maketitle
\thispagestyle{fancyplain} 

%abstracts
\begin{abstract}
\noindent
\textbf{Zusammenfassung}: Die von Hans Scharoun entworfene
Staatsbibliothek am Kulturforum ist ein zentraler Ort wissenschaftlicher
wie literarischer Textproduktion in Berlin. Neben der hohen Raumqualität
dieser Ikone moderner Bibliotheksarchitektur befördert auch ihre
charakteristische, von Wim Wenders sogar in seinem Film Der Himmel über
Berlin (1987) ästhetisierte Geräuschkulisse, ganz wesentlich die
kreative Schreibarbeit im Lesesaal. In diesem Beitrag geht es um ein in
Kooperation mit dem Hörverlag speak low sowie der
Medienwissenschaftlerin Hannah Wiemer realisiertes Projekt der
Staatsbibliothek zur akustischen Vermessung ihres Scharoun-Gebäudes am
Kulturforum. Am Vorabend von dessen Generalinstandsetzung soll damit
sowohl dem Inspirationspotential der Sounds of Stabi nachgegangen als
auch und vor allem die Möglichkeit geschaffen werden, sich die
Staatsbibliothek nach Hause zu holen -- gerade während ihrer
baubedingten Schließzeit.

\begin{center}\rule{0.5\linewidth}{0.5pt}\end{center}

\textbf{Summary}: The Staatsbibliothek at the Kulturforum, designed by
Hans Scharoun, is a central space for scientific and literary text
production in Berlin. In addition to the high spatial quality of this
icon of modern library architecture, its characteristic soundscape,
which was even aestheticized by Wim Wenders in his film Wings of Desire
(1987), also significantly promotes creative writing in the reading
room. This article discusses a project realized by the Staatsbibliothek
in collaboration with the publishing house speak low and media scientist
Hannah Wiemer to conduct an acoustic survey of the Scharoun building at
the Kulturforum. On the eve of its upcoming renovation, the aim is both
to explore the inspirational potential of the Sounds of Stabi and, above
all, to create the opportunity to take the Staatsbibliothek home --
especially during its construction-related closure.
\end{abstract}

%body
Im Rahmen der 18. Architekturbiennale von Venedig präsentierte das
international tätige Hamburger Büro \emph{gmp. Architekten von Gerkan,
Marg und Partner} eine Auswahl seiner aktuellen Projekte zum Bauen im
Bestand. Zu den sieben im Rahmen der Ausstellung \emph{Umbau -- Nonstop
Transformation} ins Rampenlicht gerückten Vorhaben zählt auch die
Generalinstandsetzung der von Hans Scharoun entworfenen Staatsbibliothek
am Berliner Kulturforum.\footnote{\url{https://umbau.gmp.de/}.} Während
die gezeigten Modelle, Zeichnungen und Renderings der übrigen Umbauten
mit Presslufthammergeräuschen oder historischen Rundfunk- und
Fernsehsendungen inszeniert wurden, erklang vor den Ansichten des
ikonischen Bibliotheksgebäudes in Dauerschleife ein raunendes Gemurmel
-- ohne jede weitere Erläuterung in Katalog, Begleittexten und Parerga.
Trotz fehlender Erklärung dürften freilich große Teile des
Ausstellungspublikums leicht die Tonspur der minutenlangen Lesesaalszene
aus Wim Wenders' Erfolgsfilm \emph{Der Himmel über Berlin} (1987)
erkannt haben. In dieser Sequenz durchschreiten die beiden in die Welt
gekommenen, für Menschenaugen aber unsichtbaren Engel Damiel und Cassiel
vor der Geräuschkulisse eines geflüsterten, diffusen Parlando unklarer
Herkunft unbemerkt den vollbesetzten Lesesaal: Werden so die tagtäglich
dort entstehenden Texte, vielleicht die Gedanken der Anwesenden hörbar,
oder strahlt dieses Klangkontinuum von den beschrifteten Buchrücken im
Regal aus, indem sie gewissermaßen den Inhalt der darin aufgestellten
Werke in den Raum projizieren? Die paradoxe Entscheidung der
Ausstellungsverantwortlichen, die Innen- und Außenaufnahmen einer
menschenleeren Bibliothek mit Stimmengemurmel beziehungsweise. -- für
die Eingeweihten -- mit Spielfilmsounds zu konnotieren, kollidiert
jedenfalls mit dem dokumentarischen Anspruch der Abbildungen und
entrückt das dargestellte Gebäude ein Stück weit der Realität.

Zwar zählen Bibliotheken wie auch Museen oder Kinos generell zu jenen
lokalisierten Utopien, gesellschaftlichen Gegenräumen und kulturellen
Projektionsflächen, die Michel Foucault als \emph{Heterotopien}
bezeichnet.\footnote{Michel Foucault: Die Heterotopien. Der utopische
  Körper. Zwei Radiovorträge, Frankfurt a.M. 2005.} Für die in zahllosen
Romanen und nicht wenigen Spielfilmen -- etwa als Konzerthaus
(\emph{Tár}) oder Flughafengebäude (\emph{Berlin Station}) --
fiktionalisierte Staatsbibliothek am Kulturforum dürfte diese
Zuschreibung aber in ganz besonderem Maße zutreffen.\footnote{Vgl.
  Martin Hollender (Hg.): \emph{Denn eine Staatsbibliothek ist, bitte
  sehr!, kein Vergnügungsetablissemang}. Die Berliner Staatsbibliothek
  in der schönen Literatur, in Memoiren, Briefen und Bekenntnissen
  namhafter Zeitgenossen aus fünf Jahrhunderten, Berlin 2008.} Dabei hat
es geradezu den Anschein, als wirke der kulturelle Überschuss, das
heterotopische Fluidum von Scharouns Bibliotheksikone als
Kreativitätsstimulans -- für die wissenschaftliche wie die literarische
Textproduktion gleichermaßen. Nach Stefanie de Velascos Einschätzung
gibt es nämlich \enquote{viele von uns hier,
Schriftstellerinnen.}\footnote{Stefanie de Velasco: Die Erste, in:
  Zitty: das Wochenmagazin für Berlin 33 (2017), S. 78.} Und Judith
Schalansky wie auch Eva Menasse bekennen sich sogar explizit zu
Inspirationspotential und Bedeutung ihres Schreibtischs im Lesesaal der
Staatsbibliothek für ihr Werk:

\begin{quote}
\enquote{Undenkbar, dass ich ohne diesen Ort auch nur eines meiner
Bücher geschrieben hätte. Es vergeht kaum eine Woche, in der ich mich
nicht hierher begebe.}\footnote{Judith Schalansky: Mein Schreibtisch
  steht in der Staatsbibliothek, in: Bibliotheksmagazin: Mitteilungen
  aus den Staatsbibliotheken in Berlin und München, 2012/3, S. 10--12;
  hier S. 10, \url{https://doi.org/10.58159/20230413-005}.}

\enquote{Meine letzten beiden Bücher sind zum größten Teil in der
Berliner Staatsbibliothek am Potsdamer Platz entstanden. So ist mir
diese Bibliothek zu einem fast mystischen Ort geworden, jenem nämlich,
wo schon zweimal etwas Großes, Schweres gelungen ist.}\footnote{Eva
  Menasse: Stabi + ich = stabil, in: Bibliotheksmagazin: Mitteilungen
  aus den Staatsbibliotheken in Berlin und München, 2013/3, S. 49--52;
  hier S. 49, \url{https://doi.org/10.58159/20230413-002}.}
\end{quote}

Da die von Hans Scharoun so bezeichnete \emph{Leselandschaft} einerseits
architektonisch konstituiert ist, sich als relationaler sozialer Raum
andererseits aber tagtäglich in Variationen neu reproduziert,\footnote{Vgl.
  dazu grundlegend Martina Löw: Raumsoziologie, Frankfurt a.M. 2000.}
dürfte ihr Kreativitätsimpuls also auch von ihrer spezifischen Akustik
ausgehen. Immerhin war es selbst Wim Wenders wichtig -- wie er im Rahmen
einer Masterclass zum Thema \emph{A Sense of Place: der Ortssinn im
Film} betont --, die Dreharbeiten zu \emph{Der Himmel über Berlin} unter
möglichst authentischen Nutzungs- und damit auch Klangbedingungen zu
realisieren, weshalb er das reguläre Publikum der Staatsbibliothek zur
Mitwirkung einlud.\footnote{Siehe
  \url{https://www.khm.de/veranstaltungen_mitschnitte/id.29797.masterclass-mit-wim-wenders/}.}
Denn entgegen ihres Namens sind Lesesäle keineswegs nur Räume der
kontemplativen, stillen Rezeption, sondern auch und ganz überwiegend
Schreibsäle, wissenschaftliche wie literarische Produktionsstätten,
Kollektive in ihr jeweiliges Schaffen vertiefter Einzelpersonen -- mit
einem charakteristischen Betriebsgeräusch. Im Fall der Staatsbibliothek
am Kulturforum wird dieses nicht zuletzt von einer markanten Haustechnik
geprägt, durchqueren doch zahlreiche Luftpoströhren und die Rollbahnen
einer kilometerlangen Kastenförderanlage mit Zischen und Rattern das
auch strukturell von funktionalen baulichen Bändern gegliederte Gebäude.

\begin{quote}
\enquote{In den Vorarbeiten zum Wettbewerb} -- so Hans Scharouns
kongenialer Büropartner Edgar Wisniewski -- \enquote{entstanden die
bibliotheksbezogenen Raumfolgen auf dem \emph{Weg des Buchs}: Poststelle
-- Akzession -- Katalogisierung -- Einbandstelle -- Magazin; hieraus
folgten die bandartigen Strukturen des Gebäudes. Das Band der Lesesäle
und die in der Mitte liegende technische Kernzone mit den
Treppenanlagen, Aufzügen, Ausgaben und Sanitärräumen u.a. waren zu der
bandartigen Struktur der bibliothekarischen Räume die logische
Konsequenz und räumliche Entsprechung. Der städtebauliche Archetyp der
Bandstadt oder Scharouns Definition des geistigen Bandes Berlins -- der
Ost-Westreihung geistiger Wirkkräfte vom Alexanderplatz bis
Charlottenburg -- war in reduzierter Dimension wohl der Urimpuls zur
Gesamtanlage.}\footnote{Edgar Wisniewski: Raumvision und Struktur:
  Gedanken über Hans Scharouns Konzeption zum Bau der Staatsbibliothek,
  in: Ekkehart Vesper (Hg.): Festgabe zur Eröffnung des Neubaus in
  Berlin: Staatsbibliothek Preußischer Kulturbesitz, Wiesbaden 1978, S.
  144--158; hier S. 144 f.}
\end{quote}

Ursprünglich hatte Scharoun sogar kommunikative Zonen in der
Leselandschaft vorgesehen -- weiterer Beleg für die radikale Modernität
seiner als scheinbar unbegrenztes, richtungsloses Raumkontinuum
konzipierten Bibliotheksvision --, abgeschirmt von speziellen
Deckenreflektoren zur Geräuschreduktion nach Plänen des renommierten
Gebäudeakustikers Lothar Cremer.\footnote{Siehe dazu ausführlich Hannah
  Wiemer: The West Berlin Staatsbibliothek and the Sound Politics of
  Libraries, in: Grey Room 87 (2022), S. 44--65,
  \url{https://doi.org/10.1162/grey_a_00343}.} Ähnlich den von Edgar
Wisniewski als Attacke auf Scharouns Ästhetik beklagten nachträglich
eingebauten Regalsystemen sollte aber auch diese Idee an den
benutzungspraktischen Vorgaben der damaligen Bibliotheksleitung
scheitern.\footnote{Vgl. Edgar Wisniewski: Hans Scharouns letztes Werk
  für Berlin: ein Bericht über den fertiggestellten Bau, in: Bauwelt 70
  (1979), S. 15--19; hier. S. 17.} Dabei ist es doch gerade die
Fähigkeit von Schallwellen, physische Barrieren zu überwinden und
durchdringen, die Scharouns Programm einer offenen, demokratischen
Bibliotheksarchitektur in direkter Nachbarschaft zur Berliner Mauer
vollendet. Und impliziert sein konzeptionelles Leitmotiv für den Entwurf
der Staatsbibliothek \emph{das Individuum in der
Gemeinschaft}\footnote{Wisniewski, Raumvision und Struktur (wie Anm. 9),
  S. 158.} nicht ein gewisses kollektives Grundrauschen?

In einer Tagebuchstudie zu Bibliotheksgewohnheiten der Stabi-Nutzenden
im Frühjahr 2020 findet ein solcher grundlegender Klangteppich
jedenfalls immer wieder Erwähnung:\footnote{Siehe dazu Romy
  Hilbrich/Barbara Heindl: Stabi 2030 -- Tagebuchstudie, Berlin 2020,
  \url{https://blog.sbb.berlin/wp-content/uploads/Tagebuchstudie_final.pdf}.}
Klappernde Tastaturen, Klick-Mäuse, Gespräche, knallende Türen, Baulärm
und vor allem Lüftungsgeräusche werden als störend und ablenkend
wahrgenommen. Das akustische Grundrauschen inklusive Papiergeraschel
wird dagegen als beruhigend beschrieben -- und die Bandbreite der
\emph{Sounds of Stabi} ist groß:

\begin{quote}
\enquote{Auch hier oben {[}\ldots{]} gibt es Geräusche. An das dumpf
ratternde Förderband (hinter einer provisorischen Wand) habe ich mich
mehr oder weniger gewöhnt. Etwas störender sind die intermittierenden,
hohen Pieptöne, die manchmal von irgendeiner Maschine ausgesendet werden
(ich weiß nicht wo, im Fernleihbereich, hinter der ehem.
Zeitungstheke?). Öfter kommt es vor, dass ein offenbar verwaistes
Diensttelefon länger klingelt. Dann fragt man sich, wann nimmt er
endlich ab?}\footnote{Hilbrich/Heindl, Stabi 2030 (wie Anm. 13),
  Tagebuch 13m, Tag 3.}
\end{quote}

Um die auditive Signatur ihres Scharoun-Gebäudes am Vorabend seiner
Generalinstandsetzung zu dokumentieren und auch während der baubedingten
Schließzeit erfahrbar zu machen, haben Staatsbibliothek und Hörverlag
\emph{speak low} die charakteristischen \emph{Sounds of Stabi} -- so der
Titel der entstandenen CD -- aufgezeichnet.\footnote{Staatsbibliothek zu Berlin, speak low, Hannah Wiemer: Sounds of Stabi. Eine akustische Vermessung der Staatsbibliothek zu Berlin am Kulturforum, Berlin 2024 (CD). \url{https://doi.org/10.58159/20231218-000}.} Zum Einsatz kam dabei
Kunstkopf-Stereophonie,\footnote{Zum Making-of siehe
  \url{https://blog.sbb.berlin/kunstkopfkino/}.} ein bereits in den
1920er Jahren entwickeltes, im Herbst 1973 im Rahmen der Internationalen
Funkausstellung und damit just zum Richtfest der Bibliothek dem
westdeutschen Publikum vorgestelltes Verfahren zur binauralen
Tonaufnahme.\footnote{Vgl. dazu
  \url{https://www.c2dh.uni.lu/projects/failure-and-success-dummy-head-recording-innovation-history-3d}.}
Allerdings fiel die Entscheidung für diese Technologie keineswegs nur
aufgrund ihres Berlin-Bezugs: Im Vergleich zu konventionellem Stereo
ermöglicht Kunstkopf-Stereophonie nämlich eine räumlichere, geradezu
immersive Hörwahrnehmung. Den Audiodateien mit den Geräuschkulissen von
neun ausgewählten Räumen und technischen Anlagen ist ein von
professionellen Stimmen gesprochener Hörspaziergang durch das Gebäude
vorangestellt -- entlang der Forschungen der Medienwissenschaftlerin
Hannah Wiemer, deren Projekt am Max-Planck-Institut für
Wissenschaftsgeschichte \emph{The Sound of Books: West Berlin's
Staatsbibliothek between Postwar City Visions, Organizational
Cybernetics, and Heterotopia} auf die Rekonstruktion der materiellen wie
intellektuellen Bedingungen der verschiedenen Klangsphären des Gebäudes
zielt.\footnote{Siehe Hannah Wiemer: West-Berliner Leselandschaft. Die
  Bibliothek als logistisches Denkwerkzeug, in: Zeitschrift für
  Medienwissenschaft 14 (2022), S. 154--161
  \url{https://doi.org/10.14361/zfmw-2022-140213}.} Daher führt die
akustische Promenade sowohl zu Foyer, Lesesaal und Cafeteria als auch in
die Eingeweide der Haustechnik -- etwa in die an die Kommandobrücke von
Raumschiff Enterprise erinnernde Umschlagzentrale der
Kastenförderanlage. Denn unter dem Eindruck der kybernetischen
Steuerungseuphorie der 1960er Jahre entwarf Hans Scharoun seine
Staatsbibliothek als gigantische selbstgesteuerte Maschine, von der
konstant zugeführte Ströme von Medien und Informationen auf dem von
Edgar Wisniewski erwähnten \emph{Weg des Buchs} gewissermaßen am
Fließband zu Ideen, Texten und letztlich zu neuen Publikationen, zu
frischem Rohstoff für den intellektuellen Produktionsprozess verarbeitet
werden.\footnote{Siehe dazu Wiemer, Leselandschaft (wie Anm. 18) und
  ihren Vortrag: Der Weg des Buches: der Scharounbau der
  Staatsbibliothek zwischen Bücher- und Straßenverkehr, via:
  \url{https://www.youtube.com/watch?v=fn3-zmepT5Q}.} Diesem
Zusammenhang in seiner Bedeutung für das akustische
Inspirationspotential von Scharouns Bibliotheksgebäude geht ein, im
begleitenden CD-Booklet veröffentlichter, Essay ausführlicher
nach.\footnote{Vgl. Christian Mathieu: Vom Band -- im Klangraum von Hans
  Scharouns Büchermaschine. Eine Erhörung, in: Sounds of Stabi. Eine
  akustische Vermessung der Staatsbibliothek zu Berlin am Kulturforum,
  Berlin 2024, S. 9--19.}

Die von \emph{speak low} realisierte Produktion umfasst zwei MP3-CDs mit
einer Gesamtlaufzeit von 557 Minuten. Die auf der ersten CD publizierten
Stereoaufnahmen können in gewohnter Weise über Lautsprecher abgespielt
werden, während die zweite CD binaurale Aufnahmen enthält. Binaurale
Aufnahmen sind dem natürlichen Hören mit beiden Ohren nachempfunden,
weshalb sie die Wahrnehmung eines räumlichen Klangbilds ermöglichen, in
dem Richtungen, Distanzen und Bewegungen von Geräuschen nicht nur
zweidimensional von links und rechts (Stereo), sondern auch
dreidimensional von vorne, hinten, oben und unten abgebildet sind. Für
binaurale Aufnahmen werden in der Regel sogenannte Kunstköpfe
eingesetzt, also Nachbildungen von menschlichen Schädeln mit
modellierten Ohrmuscheln, in deren ‚Gehörgängen' jeweils ein Mikrofon
sitzt. Das zweispurig aufgenommene binaurale Audiosignal beinhaltet
daher auch die Filterungen und Reflektionen der Ohrmuscheln sowie die
Abschattungen des Kopfes (und Torsos), die das menschliche Gehirn als
räumliche Eigenschaften von Klängen interpretiert. Damit ein
dreidimensionales Klangbild entsteht, sollte binaurales Audio über
Kopfhörer gehört werden.

Die im März 2024 mit dem Spitzenplatz der monatlichen
\emph{hr2-Hörbuchbestenliste} von Hessischem Rundfunk und Börsenblatt
ausgezeichnete\footnote{\url{https://www.boersenblatt.net/sites/default/files/documents/2024-02/hbl_2403plak.pdf}.}
sowie in der Kategorie \emph{Wortkunst} für den Preis der deutschen
Schallplattenkritik nominierte Doppel-CD ist im Buchhandel erhältlich
sowie direkt über \emph{speak low}.\footnote{\url{https://www.schallplattenkritik.de/bestenlisten/longlist/longlist-2-2024}.}
Alle Geräuschaufnahmen im Stereo- wie Binauralformat sind zusätzlich in
eine virtuelle Ausstellung auf den Seiten der Staatsbibliothek
eingebunden\footnote{\url{http://sbb.berlin/sounds}.} und stehen über
das Multimediarepositorium der Stiftung Preußischer Kulturbesitz zum
kostenfreien Download für den nicht-kommerziellen Privatgebrauch zur
Verfügung.\footnote{Staatsbibliothek zu Berlin, speak low, Hannah Wiemer: Sounds of Stabi. Eine akustische Vermessung der Staatsbibliothek zu Berlin am Kulturforum, Berlin 2024 (CD). \url{https://doi.org/10.58159/20231218-000}.}

\hypertarget{zitierte-quellen-und-forschungsliteratur}{%
\section{Zitierte Quellen und
Forschungsliteratur}\label{zitierte-quellen-und-forschungsliteratur}}

Michel Foucault: Die Heterotopien. Der utopische Körper. Zwei
Radiovorträge, Frankfurt a.M. 2005.

Romy Hilbrich/Barbara Heindl: Stabi 2030 -- Tagebuchstudie, Berlin
2020,
\url{https://blog.sbb.berlin/wp-content/uploads/Tagebuchstudie_final.pdf}.

Martin Hollender (Hg.): \emph{Denn eine Staatsbibliothek ist, bitte
sehr!, kein Vergnügungsetablissemang}. Die Berliner Staatsbibliothek in
der schönen Literatur, in Memoiren, Briefen und Bekenntnissen namhafter
Zeitgenossen aus fünf Jahrhunderten, Berlin 2008.

Martina Löw: Raumsoziologie, Frankfurt a.M. 2000.

Christian Mathieu: Vom Band -- im Klangraum von Hans Scharouns
Büchermaschine. Eine Erhörung, in: Sounds of Stabi. Eine akustische
Vermessung der Staatsbibliothek zu Berlin am Kulturforum, Berlin 2024,
S. 9--19 (Booklet zur CD).

Eva Menasse: Stabi + ich = stabil, in: Bibliotheksmagazin:
Mitteilungen aus den Staatsbibliotheken in Berlin und München, 2013/3,
S. 49--52, \url{https://doi.org/10.58159/20230413-002}.

Judith Schalansky: Mein Schreibtisch steht in der Staatsbibliothek,
in: Bibliotheksmagazin: Mitteilungen aus den Staatsbibliotheken in
Berlin und München, 2012/3, S. 10--12,
\url{https://doi.org/10.58159/20230413-005}.

Staatsbibliothek zu Berlin, speak low, Hannah Wiemer: Sounds of Stabi. Eine akustische Vermessung der Staatsbibliothek zu Berlin am Kulturforum, Berlin 2024 (CD). \url{https://doi.org/10.58159/20231218-000}.

Stefanie de Velasco: Die Erste, in: Zitty: das Wochenmagazin für
Berlin 33 (2017), S. 78.

Hannah Wiemer: Der Weg des Buches: der Scharounbau der
Staatsbibliothek zwischen Bücher- und Straßenverkehr (2021),
\url{https://www.youtube.com/watch?v=fn3-zmepT5Q}.

Hannah Wiemer: The West Berlin Staatsbibliothek and the Sound
Politics of Libraries, in: Grey Room 87 (2022), S. 44--65,
\url{https://doi.org/10.1162/grey_a_00343}.

Hannah Wiemer: West-Berliner Leselandschaft. Die Bibliothek als
logistisches Denkwerkzeug, in: Zeitschrift für Medienwissenschaft 14
(2022), S. 154--161, \url{https://doi.org/10.14361/zfmw-2022-140213}.

Edgar Wisniewski: Raumvision und Struktur: Gedanken über Hans
Scharouns Konzeption zum Bau der Staatsbibliothek, in: Ekkehart Vesper
(Hg.): Festgabe zur Eröffnung des Neubaus in Berlin: Staatsbibliothek
Preußischer Kulturbesitz, Wiesbaden 1978, S. 144--158.

Edgar Wisniewski: Hans Scharouns letztes Werk für Berlin: ein Bericht
über den fertiggestellten Bau, in: Bauwelt 70 (1979), S. 15--19.

\paragraph{Websites: } ~

\url{http://sbb.berlin/sounds}

\url{https://blog.sbb.berlin/kunstkopfkino/}

\url{https://umbau.gmp.de/}

\url{https://www.boersenblatt.net/sites/default/files/documents/2024-02/hbl_2403plak.pdf}

\url{https://www.c2dh.uni.lu/projects/failure-and-success-dummy-head-recording-innovation-history-3d}

\url{https://www.khm.de/veranstaltungen_mitschnitte/id.29797.masterclass-mit-wim-wenders/}

\url{https://www.schallplattenkritik.de/bestenlisten/longlist/longlist-2-2024}


%autor
\begin{center}\rule{0.5\linewidth}{0.5pt}\end{center}

\textbf{Insa Hansen-Goos} studierte Germanistik und Kunstgeschichte in
Hamburg, Stockholm und Potsdam. Nach dem Studium absolvierte sie ein
Volontariat im Verbrecher Verlag, zudem organisiert sie Lesungen und
Literaturveranstaltungen in Berlin. Bei speak low ist sie seit 2019
unter anderem für die Bereiche Lektorat, Vertrieb sowie Rechte und
Lizenzen zuständig.

\textbf{Barbara Heindl} (\url{https://orcid.org/0000-0002-1395-647X})
studierte Germanistik und Romanistik in Tübingen und war anschließend
wissenschaftliche Mitarbeiterin im Bereich Kulturwissenschaft an der
Europa-Universität Viadrina. Seit 2017 arbeitet sie in der
Staatsbibliothek zu Berlin und leitet nach Stationen im Fachreferat und
in der Benutzungsforschung den Bereich Presse- und
Öffentlichkeitsarbeit.

\textbf{Harald Krewer} (\url{http://d-nb.info/gnd/1017320144}) studierte
am Max Reinhardt Seminar in Wien Theaterregie und erhielt anschließend
vom deutsch-französischen Kulturrat ein Arbeitsstipendium an der Pariser
Comédie Française. Neben zahlreichen Theaterarbeiten als Regisseur und
Dramaturg in Deutschland und Österreich, arbeitet er seit 1997 als
freier Mitarbeiter in der Hörspielabteilung des Österreichischen
Rundfunks. Als Hörspielregisseur ist er für den ORF und verschiedene
ARD-Sendeanstalten tätig. Seit 2003 ist er Dozent am Wiener Max
Reinhardt Seminar und seit 2006 Mitinhaber des Hörverlags speak low.

\textbf{Christian Mathieu} (\url{https://orcid.org/0000-0002-1974-6895})
wurde nach einem Studium der Geschichte und Kunstgeschichte 2006 mit
einer Dissertation zur Umwelt- und Kulturgeschichte Venedigs in der
Frühen Neuzeit promoviert. Nach Stationen an der Herzog August
Bibliothek Wolfenbüttel sowie der Bayerischen Staatsbibliothek München
arbeitet er seit 2012 als wissenschaftlicher Bibliothekar und
Projektmanager an der Staatsbibliothek zu Berlin. Seine
Tätigkeitsschwerpunkte liegen dort auf den Feldern von Digitalisierung,
Open Access und Wissenschaftskommunikation.

\end{document}