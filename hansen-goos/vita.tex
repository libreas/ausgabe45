\begin{center}\rule{0.5\linewidth}{0.5pt}\end{center}

\textbf{Insa Hansen-Goos} studierte Germanistik und Kunstgeschichte in
Hamburg, Stockholm und Potsdam. Nach dem Studium absolvierte sie ein
Volontariat im Verbrecher Verlag, zudem organisiert sie Lesungen und
Literaturveranstaltungen in Berlin. Bei speak low ist sie seit 2019
unter anderem für die Bereiche Lektorat, Vertrieb sowie Rechte und
Lizenzen zuständig.

\textbf{Barbara Heindl} (\url{https://orcid.org/0000-0002-1395-647X})
studierte Germanistik und Romanistik in Tübingen und war anschließend
wissenschaftliche Mitarbeiterin im Bereich Kulturwissenschaft an der
Europa-Universität Viadrina. Seit 2017 arbeitet sie in der
Staatsbibliothek zu Berlin und leitet nach Stationen im Fachreferat und
in der Benutzungsforschung den Bereich Presse- und
Öffentlichkeitsarbeit.

\textbf{Harald Krewer} (\url{http://d-nb.info/gnd/1017320144}) studierte
am Max Reinhardt Seminar in Wien Theaterregie und erhielt anschließend
vom deutsch-französischen Kulturrat ein Arbeitsstipendium an der Pariser
Comédie Française. Neben zahlreichen Theaterarbeiten als Regisseur und
Dramaturg in Deutschland und Österreich, arbeitet er seit 1997 als
freier Mitarbeiter in der Hörspielabteilung des Österreichischen
Rundfunks. Als Hörspielregisseur ist er für den ORF und verschiedene
ARD-Sendeanstalten tätig. Seit 2003 ist er Dozent am Wiener Max
Reinhardt Seminar und seit 2006 Mitinhaber des Hörverlags speak low.

\textbf{Christian Mathieu} (\url{https://orcid.org/0000-0002-1974-6895})
wurde nach einem Studium der Geschichte und Kunstgeschichte 2006 mit
einer Dissertation zur Umwelt- und Kulturgeschichte Venedigs in der
Frühen Neuzeit promoviert. Nach Stationen an der Herzog August
Bibliothek Wolfenbüttel sowie der Bayerischen Staatsbibliothek München
arbeitet er seit 2012 als wissenschaftlicher Bibliothekar und
Projektmanager an der Staatsbibliothek zu Berlin. Seine
Tätigkeitsschwerpunkte liegen dort auf den Feldern von Digitalisierung,
Open Access und Wissenschaftskommunikation.
