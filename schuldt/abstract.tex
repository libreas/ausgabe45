\textbf{Abstract:} Die Frage, ob «Musikalien» wie Noten oder
Musikliteratur Teil der Volksbüchereien, also der Vorgänger der heutigen
Öffentlichen Bibliotheken, sein sollten, wurde schon ganz am Anfang des
(deutschen) Volksbüchereiwesens gestellt. Grundsätzlich wurde dies
schnell bejaht und bald angefangen, eigenständige Musikabteilungen und
Öffentliche Musikbüchereien einzurichten. Von 1900 bis 1945 erschien zu
diesem Themenkomplex in der deutschen bibliothekarischen Fachliteratur
eine überschaubare, aber doch beachtliche Zahl von Texten. Diese werden
hier in chronologischer Reihenfolge vorgestellt. Sichtbar wird dabei,
dass sich Volksbüchereien und deren Musikabteilungen gemeinsam
entwickelten, aber auch, dass diese Entwicklung immer eingelassen war in
der Entwicklung der deutschen Gesellschaft.

\textbf{Abstract:} The question of whether ``musicalia'' such as sheet
music or music literature should be part of the ``Volksbüchereien'',
i.e.~the predecessors of today's public libraries, was raised at the
very beginning of the (German) Volksbücherei movement. In principle,
this was quickly answered in the affirmative. Soon, independent music
departments and public music libraries were set up. From 1900 to 1945, a
manageable but considerable number of texts concerning this topical
field appeared in German library literature. These are presented here in
chronological order. It becomes clear that the Volksbüchereien and their
music departments developed together, but that this development was also
always embedded in the development of German society.
