\begin{center}\rule{0.5\linewidth}{0.5pt}\end{center}

\textbf{Autor*innen}

Jens Bemme

Jens Bemme studierte Verkehrswirtschaft und Lateinamerikastudien. Heute interessiert er sich für Dorfbacköfen 
und historisches Radfahrerwissen um 1900 in der Oberlausitz und der Ostseeprovinzen. Mit der ‘Datenlaube’ und 
Christian Erlinger erschließt er Wikisource-Volltexte der Illustrierten ‘Die Gartenlaube’ offen in Wikidata. 
Als Mitarbeiter der SLUB Dresden begleitet Jens landeskundliche Citizen Science-Initiativen insbesondere mit 
den digitalen Werkzeugen und Gemeinschaften der Wikimedia-Bewegung. Mastodon: JensB@openbiblio.social

(\url{https://orcid.org/0000-0001-6860-0924})

Juliane Flade

Juliane Flade studierte an der TU Dresden Sprach-, Literatur- und Kulturwissenschaften. Ihr Schwerpunkt lag hierbei 
auf der Literatur der Gegenwart im ländlichen Raum. Vor ihrem Studium arbeitete sie als Logopädin. Aktuell ist sie 
in der SLUB Dresden als Projektmanagerin für Inklusion und Citizen Science tätig. Dabei ist ihr Ziel Quellen, Wissen 
und die Bibliothek zugänglicher zu machen. Ein Beispiel hierfür ist das Projekt Gesprochene Wikisource.

(\url{https://orcid.org/0000-0002-3249-7299})

Caroline Förster

Caroline Förster ist Geschäftsführerin des Dresdner Geschichtsverein e.V.

