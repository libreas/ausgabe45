\textbf{Zusammenfassung}: Digitale Sammlungen klingen erstmal nicht. Aber dann.

`Spoken Wikisource -- German' enthält Töne von Joachim Ringelnatz,
gesprochene, Texte aus `Die Gartenlaube', auch `spoken', Stimmen und
Ideen. Der Klang der `Gesprochenen Wikisource' ist vielschichtig
beziehungsweise kann und könnte er sehr vielseitig sein -- metaphorisch
als der Klang des Projekts `Gesprochene Wikisource', durch die Stimmen
der beteiligten Gemeinschaften oder der Räume, in denen diese Aufnahmen
entstehen; zum Beispiel Tonstudios, in denen Sprecher:innen sprechen.

Gesprochen werden Textquellen aus Wikisource akustisch erfahrbar.
Wahrnehmbar wird dadurch nicht nur der Inhalt, sondern auch eine
Textinterpretation. Im Bibliothekskatalog gibt es diese Möglichkeit noch
nicht, die Quellen sind entweder als Text oder Audio verfügbar, hybrid
selten.

Beim Einsprechen und Hören ist ein Teil der Rezeption zu spüren:
Wie verstehe und interpretiere ich den Text einer Vorleser- oder
Sprecherin, die sich mit gesprochenem Volltext des Portals Wikisource
auseinandersetzt?! Dabei findet Interpretation statt, mehr als eine.

Wikisource gewinnt so als gesprochene digitale Sammlung weitere
Deutungs- und Metaebenen -- Nuancen, Zugänge und Freiheitsgrade,
Bedeutungen, Links. Dieser `Sound of Gesprochene Wikisource' ist ein
Nachhall medialer Auseinandersetzungen mit \ldots{} Text in einer
Bibliothek. Die Autorinnen berichten, betonen und spielen dabei mit
Aspekten der `Gesprochenen Wikisource' in historischen, bibliophilen und
modernen Linkzusammenhängen mit dem Podcaststudio der SLUB, mit den
offenen Kulturdaten des Dresdner Geschichtsvereins, mit `DatenlaubeJam'
am Dienstag und Lesungen im Advent -- `The Sound of Wikisource'
sozusagen, `read, spoken and linked open'.

\begin{center}\rule{0.5\linewidth}{0.5pt}\end{center}
