\documentclass[a4paper,
fontsize=11pt,
%headings=small,
oneside,
numbers=noperiodatend,
parskip=half-,
bibliography=totoc,
final
]{scrartcl}

\usepackage[babel]{csquotes}
\usepackage{synttree}
\usepackage{graphicx}
\setkeys{Gin}{width=.4\textwidth} %default pics size

\graphicspath{{./plots/}}
\usepackage[english]{babel}
\usepackage[T1]{fontenc}
%\usepackage{amsmath}
\usepackage[utf8x]{inputenc}
\usepackage [hyphens]{url}
\usepackage{booktabs} 
\usepackage[left=2.4cm,right=2.4cm,top=2.3cm,bottom=2cm,includeheadfoot]{geometry}
\usepackage[labelformat=empty]{caption} % option 'labelformat=empty]' to surpress adding "Abbildung 1:" or "Figure 1" before each caption / use parameter '\captionsetup{labelformat=empty}' instead to change this for just one caption
\usepackage{eurosym}
\usepackage{multirow}
\usepackage[ngerman]{varioref}
\setcapindent{1em}
\renewcommand{\labelitemi}{--}
\usepackage{paralist}
\usepackage{pdfpages}
\usepackage{lscape}
\usepackage{float}
\usepackage{acronym}
\usepackage{eurosym}
\usepackage{longtable,lscape}
\usepackage{mathpazo}
\usepackage[normalem]{ulem} %emphasize weiterhin kursiv
\useunder{\uline}{\ul}{}
\usepackage[flushmargin,ragged]{footmisc} % left align footnote
\usepackage{ccicons} 
\setcapindent{0pt} % no indentation in captions
\usepackage{xurl} % Breaks URLs
\usepackage{makecell}
\renewcommand{\arraystretch}{1.5}


%%%% fancy LIBREAS URL color 
\usepackage{xcolor}
\definecolor{libreas}{RGB}{112,0,0}

\usepackage{listings}

\urlstyle{same}  % don't use monospace font for urls

\usepackage[fleqn]{amsmath}

%adjust fontsize for part

\usepackage{sectsty}
\partfont{\large}

%Das BibTeX-Zeichen mit \BibTeX setzen:
\def\symbol#1{\char #1\relax}
\def\bsl{{\tt\symbol{'134}}}
\def\BibTeX{{\rm B\kern-.05em{\sc i\kern-.025em b}\kern-.08em
    T\kern-.1667em\lower.7ex\hbox{E}\kern-.125emX}}

\usepackage{fancyhdr}
\fancyhf{}
\pagestyle{fancyplain}
\fancyhead[R]{\thepage}

% make sure bookmarks are created eventough sections are not numbered!
% uncommend if sections are numbered (bookmarks created by default)
\makeatletter
\renewcommand\@seccntformat[1]{}
\makeatother

% typo setup
\clubpenalty = 10000
\widowpenalty = 10000
\displaywidowpenalty = 10000

\usepackage{hyperxmp}
\usepackage[colorlinks, linkcolor=black,citecolor=black, urlcolor=libreas,
breaklinks= true,bookmarks=true,bookmarksopen=true]{hyperref}
\usepackage{breakurl}

%meta
%meta

\fancyhead[L]{C. Frick et al.\\ %author
LIBREAS. Library Ideas, 45 (2024). % journal, issue, volume.
\href{https://doi.org/10.18452/29148}{\color{black}https://doi.org/10.18452/29148}
{}} % doi 
\fancyhead[R]{\thepage} %page number
\fancyfoot[L] {\ccLogo \ccAttribution\ \href{https://creativecommons.org/licenses/by/4.0/}{\color{black}Creative Commons BY 4.0}}  %licence
\fancyfoot[R] {ISSN: 1860-7950}

\title{\LARGE{Being queer (in libraries) is political}}% title
\author{Claudia Frick, Philipp Zeuner, Caleb Buchert,\\ Daniela Markus, Norma Fötsch, Yvonne Fischer,\\ Emma Wieseler, Sabrina Ramünke, Nik Baumann} % author

\setcounter{page}{1}

\hypersetup{%
      pdftitle={Being queer (in libraries) is political},
      pdfauthor={Claudia Frick, Philipp Zeuner, Caleb Buchert, Daniela Markus, Norma Fötsch, Yvonne Fischer, Emma Wieseler, Sabrina Ramünke, Nik Baumann},
      pdfcopyright={CC BY 4.0 International},
      pdfsubject={LIBREAS. Library Ideas, 45 (2024).},
      pdfkeywords={LGBTQIA+, queer, safe space, diversity, collection, patrons, staff, anti-discrimination, networking},
      pdflicenseurl={https://creativecommons.org/licenses/by/4.0/},
      pdfurl={https://doi.org/10.18452/29148},
      pdfdoi={10.18452/29148},
      pdflang={en},
      pdfmetalang={en}
     }


\date{}
\begin{document}

\maketitle
\thispagestyle{fancyplain} 

%abstracts
\begin{abstract}
\noindent
\textbf{Abstract}: \textit{Queerbrarians} is a German-speaking network of
queer librarians and people working or aiming to work in libraries or
library institutions that was founded in November 2023. Based upon
topics gathered at the first network meeting, this article presents the
motivations for the creation of the network as well as its ideas and
visions of more queer-friendly libraries. \textit{Queerness} is still an
underrepresented topic in libraries in German-speaking countries,
whether it's about patrons, about people working in libraries, or about
the collections or the catalogs. This needs to change. \textit{Queerbrarians} and
this article want to initiate this change, explain why it is important
and valuable, and raise awareness for the queer perspective. It is
almost inevitable that (library) policy considerations will come into
play, because being queer (in libraries) is political.
\end{abstract}

\begin{abstract}
\noindent
\textbf{Keywords}: LGBTQIA+, queer, safe space, diversity, collection, patrons, staff, anti-discrimination, networking
\end{abstract}

%body
\hypertarget{queerness-and-libraries}{%
\section{Queerness and libraries}\label{queerness-and-libraries}}

Libraries are by the people and for the people and therefore inevitably
reflect human realities, human needs, and human knowledge in all its
facets through their staff, patrons, services, and collections. This
also includes the dimensions of \emph{identity} and \emph{orientation}
(Diversity Charter n.d.; Elsheri et al.~2022, Supplement Figure 2; Timmo
D. 2022). In German the term
\enquote{Geschlechtsidentität} is usually used,
which is an insufficient translation of the more accurate English term
\href{https://lgbtqia.fandom.com/wiki/Gender\_identity}{gender \textit{identity}}
since English allows us to distinguish between sex and gender while
German doesn't. According to the
\href{https://lgbtqia.mywikis.wiki/wiki/Split_Attraction_Model}{split-attraction
model}, \emph{orientation} can be defined as romantic and sexual
orientation (Glass 2022). In this article, we combine all of these
aspects under the terms \emph{queerness} and \emph{queer} to refer to
people who identify as part of the
\href{https://lgbtqia.mywikis.wiki/wiki/LGBT}{LGBTQIA+} community and
their lived realities and needs.

We choose \emph{queerness} and \emph{queer} with care and intention not
only for this article, but also for \emph{Queerbrarians,} the network of
queer people working or aiming to work in libraries that this article is
about. The term \emph{queer} is not without controversy and was used as
an insult for a long time before the community reclaimed it as a
positive term (Diversity Arts Culture n.d.). Today, it is sometimes even
used to avoid more detailed labels (Log 2022) or as a synonym for
\emph{LGBTQIA+} (Wright 2024). However, not all members of the community
use it to identify themselves or accept it as a superordinate label.
Others go so far as to explicitly remove themselves from the community,
distancing themselves from it and the term \emph{queer}. There are two
recent and well-known examples of this in Germany. Jens Spahn is the
former Federal Minister of Health and member of the Christian Democratic
Union of Germany who immortalized himself with the words\enquote{I'm not queer, I'm gay.}\footnote{\enquote{Ich
  bin nicht queer, ich bin schwul.}} (Achterberg 2023). Alice Weidel is
a member of the Bundestag and member of the Alternative for Germany, a
right-wing political party, who proclaimed \enquote{I'm not
queer, I'm married to a woman I've known
for twenty years.}\footnote{\enquote{Ich bin nicht queer, sondern ich
  bin mit einer Frau verheiratet, die ich seit zwanzig Jahren kenne.}}
(Achterberg 2023). We have no intention of contradicting their refusal
to use the term \emph{queer} for themselves, especially since
\emph{queer} is a political term and includes the commitment against
discrimination and for equal treatment of all members of the LGBTQIA+
community, to which these people don't commit themselves
to. So the fact that we use \emph{queer} here and in the network is in
itself political.

Some readers might also be irritated by the fact that \emph{we} is being
used in this article. This is not always common in journal articles.
However, we recognize that our identification as part of the LGBTQIA+
community has an impact on both the perspectives presented here and the
visibility of queer voices in the library world. In German-speaking
countries, \emph{queerness} is still an underrepresented topic in
libraries compared to the US (Gerlach 2023). According to the loop
model, there is an interdependency between the self-image of libraries,
library science and practice, and the personal positioning of librarians
(Gerlach 2023). The underrepresentation mentioned above is therefore
also due to the reluctance of queer people involved in research and
practice to take a personal stance and voice that they are affected
themselves. As part of \emph{Queerbrarians,} we would like to positively
reinforce this interdependence both with the contribution we make with
this article and with the creation of \emph{Queerbrarians.}\footnote{We
  are writing here as part of the \emph{Queerbrarians} and not for our
  institutions and libraries.} We take our cue from those who came
before us, who paved the way for us and whose openness made others feel
less alone and allowed them to be approached with questions about name
changes, advice on coming out to colleagues and much more (Walters
2023).

\hypertarget{queerbrarians}{%
\section{Queerbrarians}\label{queerbrarians}}

The idea for \emph{Queerbrarians} was born in a small group during the
planning and submission for a hands-on lab for BiblioCon 2024 (Zeuner
et al.~2024).\footnote{The biggest library conference in Germany.
  \url{https://2024.bibliocon.de/}} The session titled \enquote{More Glitter?
How To LGBTQIA+ Safe Space for Libraries} was planned as an open
discussion for collectively answering the questions of how one can be a
safe person for the LGBTQIA+ community and make the library a safe(r)
space (The Roestone Collective 2014; Minkov 2021). It was noted with
some frustration that the impetus for the discussion and the design of
the hands-on lab was once again in the hands of the community itself.

If the LGBTQIA+ community wants change, it has to create it itself. This
is a dilemma because overburdening, i.\,e. overloading members of the
community with additional emotional and actual work around the topic of
\emph{queerness} (Heinze 2021), is not to be underestimated as an
additional burden to already existing professional activities. This was
also discussed at the first \emph{Queerbrarians} meeting\emph{:} The
additional burden of the unintentionally assumed or ascribed role as a
main contact when dealing with \emph{queerness}, advocating for
visibility and against discrimination or by coming out, mixes with the
need to finally see change.

Roberto wrote about his experiences as a
\href{https://lgbtqia.fandom.com/wiki/Transgender}{trans} man in his
library in 2011:

\begin{quote}
\enquote{Accept that you will be That Transgender Library Staffer for a while,
just because this may be new and unusual at your work, and because
people like to gossip. If you become heavily involved in workplace
activism, you may ultimately become That Angry Transgender Library
Staffer Who Ruins Everything.} (Roberto 2011, p.~127)
\end{quote}

The fact that even in 2024, people with such a role and commitment to
the LGBTQIA+ community are still not always met positively and are
rarely appreciated for their efforts is an additional complication. In
the end, the joke \enquote{Then we might as well set up a network and at least
help each other} became a reality and \emph{Queerbrarians} was founded.

The first meeting with around 60 people took place online on November
21, 2023. This was the overwhelming number of people who responded to
the call via the DACH\footnote{DACH is an acronym used for the German
  speaking countries Germany (D), Austria (A) and Switzerland (CH).}
Discord server, TikTok and several German library mailing lists. At this
meeting, communication channels through Discord\footnote{Discord
  (\url{https://discord.com/}) is a communication platform on which
  communities can organize and exchange information on their own servers
  in voice, video and text channels.} and email\footnote{If you are not
  on Discord but would still like to be informed about meetings, write
  to
  \href{mailto:librarians@queerbrarians.de}{\nolinkurl{librarians@queerbrarians.de}}
  and ask to be added to the email distribution list.} were established,
further meetings were planned and topics, wishes and concerns were
collected. Everything that interests queer library people, on a large
and small scale. In the end, the focus was still on networking: Who are
you? Where do you work? How are you doing there? What are your pronouns?
Is anyone else here
\href{https://lgbtqia.mywikis.wiki/wiki/Aroace}{aroace}? No less central
was the establishment of certain rules that apply both to the virtual
meetings\footnote{\url{https://queerbrarians.de/naechste-termine/}} and
to Discord (Table 1).

\begin{table}[h]
\begin{tabular}{|ll|}
\hline
\multicolumn{2}{|l|}{\makecell[tl]{Queerbrarians is a safe space for the queer community. To ensure that it remains a safe\\ space, there are a few rules for the Discord server and the meetings.}}                  \\ \hline
\multicolumn{1}{|l|}{1}  & Respectful and friendly interaction with one another.                                                                                                                      \\ \hline
\multicolumn{1}{|l|}{2}  & Respect the pronouns of others (click on the profile picture to see them, otherwise ask).                                                                                  \\ \hline
\multicolumn{1}{|l|}{3}  & \makecell[tl]{Discrimination on the basis of ethnicity, sex, gender identity, sexual or romantic\\ orientation, religion, nationality, age, heritage, or disability is not tolerated here.} \\ \hline
\multicolumn{1}{|l|}{4}  & Stick to appropriate content.                                                                                                                                              \\ \hline
\multicolumn{1}{|l|}{5}  & \makecell[tl]{No topics discussed during the meeting or on Discord will be made public in any\\ way unless it has been mutually agreed on beforehand.}                                      \\ \hline
\multicolumn{1}{|l|}{6}  & \makecell[tl]{Nobody is forced to come out. If you’re here, it’s assumed that you’re part of th\\e LGBTQIA+ community and that’s all anyone needs to know.}                                 \\ \hline
\multicolumn{1}{|l|}{7}  & \href{https://en.wikipedia.org/wiki/TERF_(acronym)}{TERF}s and trans hostile people in general are not tolerated here.                                                                                                          \\ \hline
\multicolumn{1}{|l|}{8}  & \makecell[tl]{The A in LGBTQIA+ does not stand for \href{https://lgbtqia.mywikis.wiki/wiki/Ally}{Ally}, but for \href{https://lgbtqia.mywikis.wiki/wiki/Asexual}{Asexual}, \href{https://lgbtqia.mywikis.wiki/wiki/Aromantic}{Aromantic} and\\ \href{https://lgbtqia.mywikis.wiki/wiki/Agender}{Agender}. You are part of the community and therefore part of us.}                                 \\ \hline
\multicolumn{1}{|l|}{9}  & No spamming and no trolling.                                                                                                                                               \\ \hline
\multicolumn{1}{|l|}{10} & No inappropriate profiles (user names, avatars, accounts and statuses).                                                                                                    \\ \hline
\multicolumn{1}{|l|}{11} & Before you send a direct message, please ask the person in question if this is welcome.                                                                                    \\ \hline
\end{tabular}
\caption{Table 1: The rules of \emph{Queerbrarians} (as of February 20, 2024).}
\end{table}

On Discord there are channels on the topic of \emph{Queerbrarianship},
i.\,e. everything to do with \emph{queerness} and libraries, as well as
general interests and the further development of \emph{Queerbrarians} as
a network. The next online meetings are also announced on the server, as
well as via email and on the website (\url{https://queerbrarians.de/}).
The meetings currently consist of a pre-defined thematic part at the
beginning, such as queer media tips or everyday life as a queer person
in your own library, and an open part for networking and community
building at the end. The current aim is to network and support each
other professionally and thematically, but also just general community
building.

\hypertarget{queer-topics}{%
\section{Queer topics}\label{queer-topics}}

At the first \emph{Queerbrarians} meeting, a digital pinboard was used
to collect topics and concerns that people would like to discuss.
Through upvotes, all attendees were also able to indicate which topics
interested them as well. In this article, we have compiled the results
and divided them into nine thematic blocks. However individual topics
may fit several different thematic blocks. We also supplement these with
suitable literature and other aspects. It's important to note that the
topics included here are not an exhaustive list, but rather a reflection
of the concerns that are of particular relevance to
\emph{Queerbrarians.}

\hypertarget{queer-togetherness}{%
\subsection{Queer togetherness}\label{queer-togetherness}}

The topic with the second most upvotes was the stated goal of \enquote{making
the library world more queer-friendly} (25 upvotes). At this point,
everyone should pause, because this can only be a goal if it is not yet
the case. This realization must be internalized by everyone, especially
the non-queer members of the library community. Only in this way is
change possible. The library world means everything and everyone:
employees, patrons, services, and collections.

Libraries cannot be safe(r) spaces for their patrons if they are not
safe(r) spaces for their employees. The topic \enquote{Queer-hostile workplaces}
was accompanied by the specific questions \enquote{How do I deal with
queer-hostile colleagues?} (12 upvotes) and \enquote{{[}W{]}hat do I do with
deadnaming in the workplace {[}...{]}?} (7 upvotes). Deadnaming refers
to the use of the name that a trans or
\href{https://lgbtqia.fandom.com/wiki/Gender_identity\#Non-binary}{non-binary}
person was given at birth, the so-called
\href{https://www.merriam-webster.com/dictionary/deadname}{deadname},
instead of the chosen name (Sinclair-Palm \& Chokly 2023). The extent to
which the latter in particular can also put a strain on professional
relationships can be seen from the comments on the question, which ask
about how to deal with incorrect forms of address in business emails and
the limits of cordiality after multiple callouts on the issue. The
question of how long one must remain friendly when boundaries are
crossed repeatedly, such as the use of the correct pronouns and form of
address, is a punch in the gut of all those who belong to one or more
marginalized groups and often experience tone policing.\footnote{In
  \enquote{tone policing}, the (allegedly inappropriate) tone of voice is
  criticized without addressing the actual argument or even explicitly
  rejecting the legitimacy of the argument with reference to the tone of
  voice, for further explanations see
  \url{https://geekfeminism.fandom.com/wiki/Tone_argument}.}

Another problem is the fact that the deadname cannot be legally changed
so easily and therefore still frequently appears on official documents
such as employment contracts, and affected people are entirely dependent
on the respect and empathy of colleagues and superiors. People who do
not use pronouns, neopronouns,\footnote{Neopronouns in German include
  \enquote{xier} or \enquote{dey}, see
  \url{https://www.nonbinary.ch/pronomen-anwendung/}.} or alternate
pronouns and forms of address are also dependent on the willingness of
their colleagues to learn new things and be flexible. Their only other
alternatives are to accept
\href{https://lgbtqia.fandom.com/wiki/Misgendering}{misgendering} or to
remain in the closet and conform to the ideas of others. This can be a
painful experience and includes, but is not limited to, people who
identify as non-binary,
\href{https://lgbtqia.fandom.com/wiki/Agender}{agender},
\href{https://lgbtqia.fandom.com/wiki/Genderqueer}{genderqueer} or
\href{https://lgbtqia.fandom.com/wiki/Genderfluid}{genderfluid} (Thorne
et al.~2020; Bradford et al.~2020). Guidelines on gender-sensitive
language that also include gender identities beyond the binary scheme
are a good start (Berufsverband Information Bibliothek 2020; Keite
2024). Unfortunately, integrating and practicing this in collegial
communication is still far from the current reality in some places.

Nevertheless, \emph{Queerbrarians} do not think about negative
experiences exclusively, but also aim to address constructive issues
such as how to \enquote{raise colleagues' awareness - for queer topics,
problems, etc.} (15 upvotes). This ranges from recognizing these small
forms of everyday discrimination to thinking about non-cis and
non-hetero perspectives and actively advocating for the LGBTQIA+
community. One example of this is \enquote{{[}u{]}npleasant Harry Potter
discussions or justifications for why I don't want to do
HP events} (8 upvotes). For all those who don't know why
queer people have a very difficult to outright negative relationship
with Harry Potter (HP), it's time to find out now by
doing your own research (Dias Correia 2023). Researching this yourself
and not expecting to have it explained to you by queer people on demand
is part of the process, the first step to becoming an
\href{https://lgbtqia.mywikis.wiki/wiki/Ally}{Ally}, and avoids the
overburdening which has already been mentioned. Creating awareness for
queer issues does not necessarily mean dismissing such an event
entirely, but being sensible of the problem, addressing it openly, not
following up if someone refuses the assignment and not scheduling one of
the openly queer people in the team to host it.

The library community needs to talk about how to create a queer-friendly
environment for staff and patrons. Establishing a \enquote{Code of Conduct for
Events} (3 upvotes) in our libraries, as is the case with formats such
as Jugend hackt\footnote{Jugend hackt is a non-profit program of the
  non-profit associations Open Knowledge Foundation Deutschland e.\,V. and
  mediale pfade -- Verein für Medienbildung e.\,V. With the motto \enquote{Improve
  the world with code}, it is aimed at young people between the ages of
  12 and 18. For the Code of Conduct, see
  \url{https://jugendhackt.org/code-of-conduct/}.}, can be a start.
Codes of conduct at our conferences and meetings, and in our libraries
in general can also help us all to reduce uncertainty and build trust in
our interactions with one another. In addition, clear position papers
are a tool for libraries to signal to members of the LGBTQIA+ community
that their actions will reflect their words. It needs \enquote{position papers
when things get serious again} (9 upvotes). The fact that things can get
serious is being felt globally and also in German-speaking countries
(Siggelkow \& Reveland 2023). Libraries should therefore be visible and
clear allies for the diversity of all identities and orientations.
Equally, this can be understood to mean that \emph{Queerbrarians}, as a
community of queer library people, want to speak out through position
papers on queer library issues. Together we can give ourselves a voice.

\hypertarget{queerness-on-site}{%
\subsection{Queerness on site}\label{queerness-on-site}}

The place in which this togetherness is created is inevitably linked to
the concept of safe(r) spaces. In addition to gender-sensitive language
and queer-friendly interaction, we also need spaces in which queerness
is taken into account. \enquote{Unisex toilets in libraries} (22 upvotes) remain
a current and important issue and have been implemented too rarely to
this date.\footnote{As a positive example, we would at least like to
  mention the all-gender toilets in the libraries of Cologne University
  of Applied Sciences and Neu-Ulm University of Applied Sciences.
  Retrieved February 20, 2024, from
  \url{https://www.asta.th-koeln.de/ueber-uns/lgbt/} and
  \url{https://www.hnu.de/alle-news/detail/2023/11/9/erste-all-gender-toiletten-an-der-hochschule-neu-ulm}.}
Just as we create a safe(r) space in the communicative space through
gender-sensitive language in which we include all identities, we can and
must also do this in the physical space.

Public libraries in particular create visibility for queer issues by
using \href{https://www.hrc.org/resources/lgbtq-pride-flags}{pride
flags} during Pride Month or permanently to highlight queer literature.
While this can be Rainbow-Washing\footnote{\enquote{Rainbow-washing} refers to
  strategies that advertise (alleged) support for the LGBTQIA+ movement
  in order to appear modern, progressive and tolerant, without actually
  implementing any measures. See also
  \url{https://thisisgendered.org/entry/rainbow-washing/}.} (Fille
2022), it still signals that queer people are welcome. The point of all
this is not to be perfect from the start, but to show that the topic is
being reflected upon and that one is open to feedback and improvement.

\hypertarget{queer-it}{%
\subsection{Queer IT}\label{queer-it}}

While we have the opportunity to correct our words and actions in direct
conversation, such as when misgendering a person, this is not possible
on the website or in registration or similar forms, such as the
registration form for new patrons (Frick \& Honold 2022). Registration
forms can usually still be changed easily, in contrast to the underlying
IT systems. Another relevant topic is therefore the connection between
\enquote{queerness and IT} (12 upvotes).

The majority of processes in libraries are now handled digitally in
predefined workflows. Everything starts with the registration form and
adding new patrons into the library's patron
administration. When discussing the topic, a uniform picture emerged:
gender is queried at these points \enquote{because the system asks for it}. The
discussion about other reasons for this leads to statistics, on the
basis of which literature is purchased and provided. However, this is
counteractive to the desire for libraries not to reproduce outdated
gender roles and to offer readers what they want to read, regardless of
their gender identity (Leyrer 2014).

Adapting or omitting the gender query and changing existing forms and
systems makes them more queer-friendly, because non-binary and trans
people in particular can be put in awkward situations by such queries.
Adjustments of this kind often require major consultations and sometimes
a legal push. However, even the current legally prescribed option in
Germany \emph{diverse} will not solve this problem, as it is only aimed
at \href{https://lgbtqia.mywikis.wiki/wiki/Intersex}{inter*} people
(Antidiskriminierungsstelle des Bundes n.d.). Here too,
\emph{Queerbrarians} want to constructively discuss how to initiate
change and find solutions, at least in libraries.

In addition to the handling of patron data, there are also discussions
about how and whether established software and processes can be
redesigned when it comes to employee data. Dealing with \enquote{specifications
and {[}overcoming{]} difficulties in the implementation of e.\,g. email
signatures and the like} (11 upvotes) is an important aspect. Among
other things, this refers to the voluntary inclusion of pronouns in
one's own email signature. A step that can theoretically
be implemented quickly and leads to the active inclusion of
one's own pronouns being normalized, as it takes the
focus off those who do so in order to avoid being misgendered (Frick \&
Honold 2022). It also helps to address others properly and avoids
confusion and uncertainty. Unfortunately, it is not uncommon to hear
reports that such information is not welcome in the signature or even
prohibited.

\hypertarget{queer-cataloging}{%
\subsection{Queer cataloging}\label{queer-cataloging}}

\emph{Queerness} must also be considered in the traditional library
field of cataloging. The topic \enquote{Queer Cataloging - which subject
heading(s) to use for queer literature?} (17 upvotes) not only came up
at the first \emph{Queerbrarians} meeting, but is a regular topic of
discussion. For example, it has already been discussed whether the
keyword \emph{LGBTQIA+} is sufficient if one medium is actually
specifically about bisexuality and another about aromanticism, or
whether it is even appropriate to tag media as \emph{diverse} or
\emph{queer} (Brown 2020; Drabinski 2013; Wilk \& Vincent 2018). The
GND\footnote{\enquote{The Integrated Authority File (GND) is a service
  facilitating the collaborative use and administration of authority
  data.}
  \url{https://www.dnb.de/EN/Professionell/Standardisierung/GND/gnd_node.html}}
catalog for subject headings has some shortcomings when it comes to
queer subject headings. Terms such as \emph{cisgender} are not listed,
unlike \emph{transgender. Gender} as the English term to describe gender
identity and not sex assigned at birth has only been implemented
recently (Aleksander 2022). The existence or non-existence and the
practice of assigning subject headings have not only recently come under
criticism (Sparber 2016).

How much depth and range we allow in our subject headings and
classifications has an impact both externally and internally. Continuous
reflection, taking into account the perspectives of those affected and
critically questioning and changing previous practices (Hutchinson \&
Nakatomi 2023) are a step in the right direction. Homosaurus
(\url{https://homosaurus.org/}) and the Queer Metadata Collective
(\url{https://queermetadatacollective.org/}) are two examples of such
initiatives. This is directly followed by considerations on \enquote{Naming in
the GND} (8 upvotes). Whether, and if so, how, a
person's dead name should be recorded there requires a
queer librarian's perspective and a discussion of
library ethics on weighing up the wishes of the person, library practice
and the sometimes outdated rules and regulations written without a queer
perspective. Since German is a very gendered language, job titles are
also clearly gendered male or female. This is particularly problematic
if they are used in an authority record for a non-binary or agender
person (Bargmann 2022).

\hypertarget{queer-collections}{%
\subsection{Queer collections}\label{queer-collections}}

The collections of libraries reflect the realities of human life, human
needs, and human knowledge in all its facets. Each type of library does
this in its own unique way, and each library is unique in its own way.
From strategic collection management to individual acquisition
decisions, many levels can influence the content represented and the
respective range and depth. External factors are also increasingly
trying to influence library collections. This has been particularly
evident in school libraries in the USA since around 2021, where
so-called book bans have dramatically expanded the scope of censorship
(Orsborn 2022). So far, there is no sign of a decline in the trend; on
the contrary, the number of banned and questioned media continues to
rise. PEN America observes and documents what is happening and notes
that it often affects especially those books that have long had to fight
for a place on the shelf: books by BIPoC (Black, Indigenous, and People
of Color) or members of the LGBTQIA+ community as well as books that
address racism, sexuality, gender, and history regardless of the
author.\footnote{\url{https://pen.org/issue/book-bans/}} Protests by
parents or initiatives, administrative decisions, or political pressure
lead to restricted access to these books or them disappearing from
school libraries altogether. The fact that pressure is also exerted on
library collections in German-speaking countries is nothing new
(Laudenbach 2023, Mobile Beratung gegen Rechtsextremismus Berlin 2023).
\emph{Queerbrarians} would therefore like to address and keep an eye on
the \enquote{banning of queer books in America - (and Germany? Hopefully
not...)} (8 upvotes), also to support the libraries abroad.

A general discussion about \enquote{Queer themes/characters in movies, series,
books} (2 upvotes) can and should also take place. In order to ensure
adequate representation of all gender identities and orientations,
creative collection development is still required in some cases. For
example, self-publishing plays an important role, especially for
underrepresented parts of the LGBTQIA+ community (Kennon 2021).
\enquote{Activist literature and associations} (2 upvotes) in libraries is also
a topic that should be addressed, as well as the
\enquote{\href{https://www.youthoutright.org/articles/fetishization-of-the-queer-community}{fetishization}
of BL/GL (in manga)} (6 upvotes). BL stands for \emph{Boys Love} and GL
for \emph{Girls Love}. It was further explained: \enquote{Choosing
non-fetishizing books is sometimes hard, especially in the mainstream
and also because it often just feels like it's the most
read material. I like to try and explain why these books/portrayed
relationships are problematic, but often find it like pushing a boulder
up a hill. Do you have problems like this too?} \emph{Queerbrarians}
want to and should be a place for precisely these kinds of questions,
which often go unheard in our own libraries. So it's no
wonder that \enquote{convincing colleagues to view {[}and{]} expand queer
collections} (7 upvotes) is also a topic.

Libraries can \enquote{normalize being queer more} (12 upvotes) through their
collections and thus realize a wish expressed by \emph{Queerbrarians.}
We do not want to ignore the fact that there are already (albeit not
always librarian) role models for libraries and collections with a queer
focus. Examples include the Queer Library (\url{https://queerbib.de/}),
the Queerfeminist Library of the General Students' Committee of the FU
Berlin (\url{https://astafu.de/bibliothek}) and the library of the Gay
Museum in Berlin (\url{https://www.schwulesmuseum.de/bibliothek/}).

\hypertarget{access-to-queer-topics}{%
\subsection{Access to queer topics}\label{access-to-queer-topics}}

\begin{quote}
\enquote{Oftentimes, when an individual is discovering and exploring their
identity, they will search for mirror characters: examples of themselves
in media as a way to understand what it means to identify a particular
way.} (Allen 2022, p.~3).
\end{quote}

Young people in particular therefore associate reading with the
experience of finding themselves in fictional characters and identifying
with them. Accompanying them on their journeys through life, difficult
situations and finding their identity can be an integral part of growing
up and finding one's own identity. For queer young
people in particular, studies show \enquote{that LGBTIQ+ identity development
processes are their primary motivators to read} (Wexelbaum 2019,
p.~115). With a greater range and density of diversity in all media
today, it has also become easier for marginalized groups to find stories
that reflect them while learning to understand their identity,
relationships, and the world around them. However, this is only possible
if access to the media is not restricted or prevented altogether by
initiatives such as book bans.

People who feel they belong to the LGBTQIA+ community experience more
bullying, sexual violence, and mental health problems (Orsborn 2022). It
is therefore all the more important that libraries act as safe(r) public
spaces and give queer people the opportunity to inform themselves,
exchange ideas, and develop in a protected environment (Wright 2024).
This can be essential, especially for people with underrepresented
identities and orientations:

\begin{quote}
\enquote{The validation and affirmation of
\href{https://lgbtqia.fandom.com/wiki/Asexual}{asexuality} as an
orientation and the equitable recognition of the full spectrum of
asexuality are particularly significant for questioning,
\href{https://lgbtqia.fandom.com/wiki/Asexual}{ace}, and
\href{https://lgbtqia.fandom.com/wiki/Asexual_spectrum}{acespec} young
readers seeking representation and who might not have encountered
inclusive and respectful stories about their experiences and
identities.} (Kennon 2021, p.~19; links added by the authors)
\end{quote}

On the occasion of Christopher Street Day (CSD), the German name for
Pride parades, in the year 2022, Hanover City Library displayed
information flyers on the topic of
\href{https://trans.fandom.com/wiki/Binding}{binding} and
\href{https://trans.fandom.com/wiki/Tucking}{tucking} (Becker 2023). The
outrage on social media was huge \enquote{and the first reaction was to simply
put the flyers away}, said Tom Becker, Head of Hanover City Library
(Mobile Beratung gegen Rechtsextremismus 2023, p.~37). Instead, he sums
up: \enquote{We need to become more resilient here -- even when it comes to
issues that are not always immediately obvious to our employees.}
(Mobile Beratung gegen Rechtsextremismus 2023, p.~37) This is how
normalization of access to queer and especially health-related queer
topics works. Libraries can make a significant contribution to the
health education and safety of queer young people, especially by
displaying such flyers. The LGBTQIA+ community has a lot of experience
in organizing and providing health information independently, which grew
out of the sad reality of necessity, and focuses on the collective
knowledge and experiences of its members (Kitzie et al.~2023). Public
library initiatives should therefore work with the community, as Hanover
City Library has done, and not give in to pushbacks.

The event \enquote{Wir lesen euch die Welt, wie sie euch gefällt}\footnote{\enquote{We
  read the world to you as you'd like it to be.}}
organized by Munich City Library in June 2023 (Heudorfer 2023) also
sparked heated discussions and demonstrations. This was a reading in
which a
\href{https://dictionary.cambridge.org/de/worterbuch/englisch/drag-queen}{drag
queen} and a
\href{https://dictionary.cambridge.org/de/worterbuch/englisch/drag-king}{drag
king} read from children's books and did educational
work. The stories were about overcoming gender stereotypes (Miebach
2023) and discovering one's own independent development
(Heizereder 2023). This reading was not the first of its kind in this
library, but it was the first with election-related and media-effective
dissenting voices from all directions. For example, the CSU (Christian
Social Union in Bavaria) spoke of \enquote{woke early sexualization} (Miebach
2023) and someone commented in a \emph{letter to the editor} for the
professional journal of Berufsverband Information Bibliothek (BuB) that
\enquote{libraries should not overload their programmes with social messages}
(Werner 2024, p.~21). Ackermann, the director of Munich City Library,
emphasized the opposite: \enquote{We need role models who show us that it is
okay to be different} (Miebach 2023).

In order to avoid the impression that access to queer topics is only
relevant for public libraries, we would like to highlight that a strong
academic library with extensive and diverse collections, supportive and
helpful staff, and relevant services in cooperation and exchange with
queer networks on campus can greatly contribute to making the campus
more inclusive for students from the LGBTQIA+ community (Clarke 2011,
Wright 2024). In addition, prospective and current librarians also need
access to queer issues. How often these topics arise and are dealt with
in professional and higher education institutions can only be speculated
upon. The choice of queer topics for theses and their publication can be
an indication and at least signals openness in university settings to
the outside world. The same applies to student projects (Berends et al.
2023, 2024). We must not underestimate how important the visibility of
\emph{queerness} in our professional field can be for those interested
in training and studying. Where we do not see \emph{queerness}, we
cannot be sure that our \emph{queerness} is welcome. But
\emph{queerness} can and should also be a regular topic in training and
studies. Mehra already described possible approaches in 2011. In
addition to formal representation, such as through working groups and
networks, official contacts and workshops to raise awareness can not
only be signals, but can also make our colleagues more sensitive or even
more diverse (Mehra 2011, Table 1).

\hypertarget{out-in-work-training-and-study}{%
\subsection{Out in work, training and
study}\label{out-in-work-training-and-study}}

At the first meeting of \emph{Queerbrarians,} the question of who is out
at work was asked at one point in casual conversation. The result was
very mixed. At this point, it is often argued that cis and hetero people
are not out at work or do not show off their identity and orientation.
However, we live in a cis-, \href{https://lgbtqia.fandom.com/wiki/Allo}{allo}- and heteronormative world. Many people are
apparently not aware that the male perceived colleague who talks about
the weekend with his wife and children does not necessarily have to be
cis and hetero. The male perceived colleague who talks about the weekend
with his husband and children however automatically outs himself as at
least not hetero, although he does nothing different from the colleague
before: talking about his weekend with his family. A non-binary person
who does not want to come out at work is inevitably misgendered because
the prevailing German language still prefers binary forms, meaning that
people primarily use male or female pronouns and forms to address others
and choose which ones to use based on their perception of the person. If
the person also has a chosen name that they do not use in order to avoid
coming out, this results in deadnaming. The opposite can also be the
case: If a non-binary or trans person has chosen a name that others feel
does not match their perceived gender, this person is often forced to
come out, justify their choice of name or sometimes have unpleasant
discussions on the subject. This is particularly the case if the
deadname still appears in official correspondence, i.\,e. their own email
address or in other places. It still has to be legally stated
everywhere, and administration and IT do not have to make any
adjustments without a legal change.

\enquote{Out at work, yes{[},{]} no? How? Dealing with a lack of understanding
and prejudice} (14 upvotes) is therefore a key aspect for the well-being
and impartiality of queer people in the workplace (Riggle et al.~2017).
According to a study from 2020, 31\,\% of respondents in Germany are not
open about their \emph{queerness} in the workplace (Vries et al. 2020).
At this point, it must be recognized that there are significant
differences within the LGBTQIA+ community. According to a study
conducted by the British government in 2018, 29\,\% of cis people and 38\,\%
of trans people surveyed were not out to anyone at work (Government
Equalities Office 2018, p.~139 and p.~142). These figures vary greatly
depending on sexual orientation. While only 18\,\% of cis people who
identify as gay or lesbian have not come out to anyone at work, 77\,\% of
cis people who identify as asexual have not. For trans people, these
figures are 27\,\% for trans people who identify as gay or lesbian and
57\,\% for trans people who identify as asexual. With regard to school,
training, and studies, it can be seen that 10\,\% of both cis and trans
respondents are not out to their classmates or fellow students
(Government Equalities Office 2018, p.~111 and p.~113). The differences
depending on sexual orientation are significantly smaller in these
groups of people, but still exist. There are no studies with this level
of detail in Germany to date.

For those that are out of the closet outside of work, training, or
studies, but not within, find themselves constantly on guard: \enquote{I was
female-identified at work, and some flavor of transgender almost
everywhere else; as I've never really been able to
completely separate my personal and professional lives, this was
incredibly difficult to do.} (Roberto 2011, p.~124) This costs energy
and puts a strain on mental health.

The desire and need for more visibility and representation must be met
with structural and community changes. This task should not be put upon
the members of the LGBTQIA+ community and certainly not be defined as
their individual responsibility. Being open about one's
\emph{queerness} still has negative consequences. \enquote{One student was told
by a senior lecturer that talking about their asexuality in their work
would limit their career.} (Benoit \& de Santos 2023, p.~13). It is just
as unacceptable that people cannot come out because they fear
repercussions as it is unacceptable to expect people to come out in
order to make \emph{queerness} more visible and normalize it. We are
talking about very personal and individual choices here, as well as
personal safety and visibility, and all options are legitimate and to be
supported.\footnote{For this reason, we have decided to collectively
  refrain from using our pronouns in order to protect the privacy of the
  individual participants.} This also means that no one should ever out
another person.

Libraries need to create an atmosphere in which all members of the
LGBTQIA+ community feel safe enough to freely decide whether or not to
come out. Employees should feel free to bring their authentic selves to
the workplace (Wright 2024). In an ideal world, if we can dream here, no
one would assume anything about another person's
identity and orientation, thus no one or everyone would need to
\enquote*{come out}. However, our cis- and heteronormative world is not
there yet. \emph{Queerbrarians} therefore also want to \enquote{normalize being
queer more} (12 upvotes) by standing together and thus being visible as
a group and not necessarily as individuals.

\hypertarget{queer-personnel-development}{%
\subsection{Queer personnel
development}\label{queer-personnel-development}}

For those who are out at work, the topic of \emph{queerness} in their
professional life quickly becomes a matter of the heart. In many cases,
this leads to queer issues and development processes getting stuck with
these people and quickly resulting in overburdening. To ensure that this
does not happen and that the necessary efforts can be spread across many
shoulders, there must be more training opportunities and educational
work on diversity, which includes \emph{queerness} among many other
topics. The need to talk about \enquote{raise colleagues' awareness - for queer
topics, problems, etc.} (15 upvotes) reflects this desire. On the other
hand, there is the problem that many employees see no need for further
training in this area (Mefebue 2016). There may be no malicious intent
behind this. \enquote{{[}the{]} workplace {[}is{]} not queer-hostile at all, but
still very heteronormative} (19 upvotes). However, ignorance on the part
of non-queer, and in some cases queer, employees means that issues that
need to be addressed are not seen and therefore not addressed. This is
where further training measures and increasing the attractiveness of
libraries as a workplace for queer people can come in.

In addition to the personal development of employees, change must also
take place at the institutional level if the situation is to change in
the long term:

\begin{quote}
\enquote{We must put our money where our mouths are. We must have leadership
that is willing to engage in brave, difficult conversations that
interrogate the hiring practices of their organizations, as well as how
to retain talented people from underrepresented backgrounds}
(Stringer-Stanback \& Jackson 2023, p.~463).
\end{quote}

One person reported that their own library is currently working on
drawing up an \enquote{equality plan} (6 upvotes). Suggestions and ideas for
queer measures that could be included in this plan were collected on the
Discord server. Such initiatives are one of many cornerstones for the
future-oriented development of libraries and their employees, in which
everyone is considered and involved.

\hypertarget{queer-network}{%
\subsection{Queer network}\label{queer-network}}

\begin{quote}
\enquote{In my working life as a library technician {[}...{]}, being trans has
led to hilarious and awkward conversations with colleagues, moments of
genuine connection, and exciting professional development.
I've also felt the impact and toll when transgender
issues enter workplace discussions. My experience of being a trans
library technician has been positive overall, but there is always fear.}
(Nault 2023, p.~46)
\end{quote}

In the network, \emph{Queerbrarians} want to share positive and negative
experiences with each other to support one another in order to not feel
alone. At the same time, this creates connections and synergies that can
create something new, like this article. It is therefore not surprising
that \enquote{Get to know new people} (28 upvotes) had the most upvotes. But
other networking activities also came up, such as mutual \enquote{book tips,
queer novels, good picture books, gender-sensitive sex education books
etc.} (11 upvotes), the \enquote{collection of all academic work from the
network {[}...{]} on the topic - e.\,g. articles, bachelor theses,
dissertations...} (6 upvotes) as an open Zotero group and an examination
of differences in the community itself, such as the \enquote{age gap between
different
LGBTQIA/\href{https://en.wikipedia.org/wiki/FLINTA*}{FLINTA}\footnote{FLINTA
  is used in German to refer to female, lesbian, inter*, non-binary,
  trans and agender people.} generations} (7 upvotes). These were all
ideas that came up at the first meeting and are currently being
addressed by the network. In all of this, it resonates that change can
best be achieved as a team and that \emph{Queerbrarians} offers a
possible framework for this.

\hypertarget{queer-future}{%
\section{Queer future}\label{queer-future}}

The representation of \emph{queerness} in libraries in German-speaking
countries, whether it's for patrons, people working in libraries, in
collections or catalogs, is still not sufficient to support queer people
and give them enough security to exist and develop freely. Based on the
experiences and wishes gathered from \emph{Queerbrarians}, and the
literature cited, we identify the still insufficient awareness of these
issues and queer lived realities in the library community as one of the
main causes. We recognize that for some it is still unusual to think
about non-cis and non-hetero perspectives, but at the same time queer
people always think about cis and hetero perspectives and cannot be
solely responsible for ensuring that queer realities are represented and
supported in library work. This is a joint mission that must not fail
due to a lack of willingness on the part of non-queer employees and
superiors.

Arguments that present the topic as a marginal issue fail to recognize
the library's mission and the fact that inclusion is
always important, even if it only affects one person. If libraries want
to be open to all realities of life, queer people must be included,
regardless of whether they are patrons or library staff.
\emph{Queerness} should not be pushed out of the conversation due to a
lack of awareness or by right-wing movements. Queer people are part of
our patrons and part of the library community. No person working in a
library should have to realize \enquote{how deeply alienating and dehumanizing
it is to always be thinking about how to better serve a community when
it's politically toxic to even acknowledge that you and
people like you are part of that community.} (Baker 2023, p.~158) We
would therefore prefer not to argue with statistics at this point, but
since some people need them, we would like to point out that according
to the Ipsos survey on LGBT+ Pride 2023, 13\,\% of the population surveyed
in Switzerland and 11\,\% in Germany identify as part of the LGBTQIA+
community (Ipsos 2023). The fact that libraries should contribute to the
normalization of queer realities cannot be argued away.

\emph{Queerbrarians} would like to actively work on formulating what
queer-friendly libraries can look like for staff and patrons and how
this can be implemented in library policy, organize training courses and
lectures, and offer open meetings for all library staff and those
working in library institutions, including non-queer people. At the same
time, we recognize our own bias as a predominantly white network --- an
overrepresentation that the entire library community must also become
aware of in order to eliminate it. Intersectional perspectives need more
space and visibility. We need more understanding for each other. We need
more knowledge about queer issues. We need more knowledge about romantic
and sexual orientations. \enquote{Wir brauchen mehr Wissen über geschlechtliche
Vielfalt, mehr Informationsmöglichkeiten und eine ehrliche
Auseinandersetzung.}\footnote{\enquote{We need more knowledge about gender
  diversity, more information options and an honest debate.}} (Lieb
2023, p.~88)

\hypertarget{english-translation}{%
\section{English translation}\label{english-translation}}

Please note that this is a translation of a German text that reflects
the cultural context of Central Europe and is limited to the latter.

This English translation was partially carried out by using the website
DeepL (\url{https://www.deepl.com/translator}) to create a basis to work
off and to speed up the translation process. The automatic translation
was checked and revised by some of the original authors, to assure its
integrity and to ensure that the message we are trying to convey did not
get lost.

We also thank Ryan Wright and Caleb England, both native English
speakers, for finalizing the English translation, as well as an
anonymous librarian and English native speaker for their support on
library-specific terminology.

\hypertarget{references}{%
\section{References}\label{references}}

Achterberg, B. (2023, November 15). Jens Spahn distanziert sich vom
Queer-Begriff: Bin nicht queer, ich bin schwul. \emph{Neue Zürcher
Zeitung}. Retrieved February 20, 2024, from
\url{https://www.nzz.ch/international/jens-spahn-distanziert-sich-vom-queer-begriff-bin-nicht-queer-ich-bin-schwul-ld.1765767}

Aleksander, K. (2022). Antrag zur Aufnahme des Sachbegriffs
\enquote{Gender} in die Gemeinsame Normdatei (GND) der Deutschen
Nationalbibliothek (DNB). \emph{027.7 Zeitschrift Für Bibliothekskultur
/ Journal for Library Culture}, \emph{9}(4).
\url{https://doi.org/10.21428/1bfadeb6.c232c54d}

Allen, M. (2022). \enquote{In a Romantic Way, Not Just a Friend Way!}: Exploring
the Developmental Implications of Positive Depictions of Bisexuality in
Alice Oseman's Heartstopper. \emph{Journal of
Bisexuality}, 23(2), 197-228.
\url{https://doi.org/10.1080/15299716.2022.2153191}

Antidiskriminierungsstelle des Bundes. (n.d.). \emph{Frau -- Mann -
Divers: Die \enquote{Dritte Option} und das Allgemeine
Gleichbehandlungsgesetz (AGG)}. Antidiskriminierungsstelle des Bundes.
Retrieved February 20, 2024, from
\url{https://www.antidiskriminierungsstelle.de/DE/ueber-diskriminierung/diskriminierungsmerkmale/geschlecht-und-geschlechtsidentitaet/dritte-option/Dritte_Option.html}

Baker, A. E. (2023). Holding onto Dreams. In K. K. Adolpho, S. G.
Krueger, \& K. McCracken (Eds.), \emph{Trans and gender diverse voices
in libraries} (pp.~157-167). Library Juice Press.

Bargmann, M., Blumesberger, S., Gruber, A., Luef, E., \& Steltzer, R.
(2022). Sacherschließung geschlechtergerecht?! Rückblick auf den
Online-Workshop am 11. Mai 2022 und Aufruf zu gemeinsamen Aktivitäten.
\emph{Mitteilungen der Vereinigung Österreichischer Bibliothekarinnen
und Bibliothekare}, \emph{75}(2), Article 2.
\url{https://doi.org/10.31263/voebm.v75i2.7582}

Becker, T. (2023, July 13). »Der Sinn von Politik ist Freiheit«.
\emph{BuB - Forum Bibliothek und Information}. Retrieved February 20,
2024, from
\url{https://www.b-u-b.de/detail/der-sinn-von-politik-ist-freiheit}

Benoit, Y., \& de Santos, R. (2023). \emph{Ace in the UK Report.}
Stonewall. Retrieved February 20, 2024, from
\url{https://www.stonewall.org.uk/system/files/ace_in_the_uk_report_2023.pdf}

Berends, A., Fabian, L. S., Fels, R., Lächelt, A.-L., Nguyen Thu, M.,
Reinhard, J., Schliemann, T., Schmidt, L., Steinike-Kuhn, J., Storch,
J., \& Waldorf, M. (2023). \emph{Diversität in Bibliotheken - Wie gut
sind Bibliotheken aufgestellt?} \url{https://doi.org/10.25968/OPUS-2933}

Berends, A., Fabian, L. S., Fels, R., Lächelt, A.-L., Nguyen Thu, M.,
Reinhard, J., Schliemann, T., Schmidt, L., Steinike-Kuhn, J., Storch,
J., \& Waldorf, M. (2024, January 17). ). Diversität in Bibliotheken --
Wie gut sind Bibliotheken aufgestellt? \emph{BuB - Forum Bibliothek und
Information}. Retrieved February 20, 2024, from
\url{https://www.b-u-b.de/detail/diversitaet-in-bibliotheken-wie-gut-sind-bibliotheken-aufgestellt}

Berufsverband Information Bibliothek (2020). \emph{Leitfaden für
gendersensible Sprache und diskriminierungsfreie Kommunikation}.
Retrieved February 20, 2024, from
\url{https://www.bib-info.de/fileadmin/public/Dokumente_und_Bilder/BIB-Standpunkte/LeitfadenSpracheBild_20201107.pdf}

Bradford, N. J., Rider, G. N., Catalpa, J. M., Morrow, Q. J., Berg, D.
R., Spencer, K. G., \& McGuire, J. K. (2020). Creating gender: A
thematic analysis of genderqueer narratives. In Joz, M., Nieder, T., \&
Bouman, W. (Eds.), \emph{Non-binary and genderqueer genders}
(pp.~37-50). Routledge.

Brown, A. (2020, August 26). How Labeling Books as \enquote{Divers} Reinforces
White Supremacy. Lee \& Low Blog. The Open Book Blog.
\url{https://blog.leeandlow.com/2020/08/26/how-labeling-books-as-diverse-reinforces-white-supremacy/}

Charta der Vielfalt (n.d.). \emph{Vielfaltsdimensionen - Die sieben
Dimensionen von Vielfalt.} Retrieved February 20, 2024, from
\url{https://www.charta-der-vielfalt.de/fuer-organisationen/vielfaltsdimensionen/}

Clarke, K. L. (2011). LGBTIQ Users and Collections in Academic
Libraries. In E. Greenblatt (Ed.), \emph{Serving LGBTIQ library and
archives users: Essays on outreach, service, collections and access}
(pp.~81-93). McFarland \& Company, Inc., Publishers.

Drabinski, E. (2013). Queering the Catalog: Queer Theory and the
Politics of Correction. \emph{The Library Quarterly}, 83(2), 94-111.
\url{https://doi.org/10.1086/669547}

Dias Correira, J. (2023). Asking the Bigger Questions: The Problem with
LGBT+ Allyship in Libraries. In K. K. Adolpho, S. G. Krueger, \& K.
McCracken (Eds.), \emph{Trans and gender diverse voices in libraries}
(pp. 449-453). Library Juice Press.

Diversity Arts Culture (n.d.). Queer. In \emph{Diversity Arts Culture}.
Retrieved January 3, 2024, from
\url{https://diversity-arts-culture.berlin/woerterbuch/queer}

Elsherif, M. M., Middleton, S. L., Phan, J. M., Azevedo, F., Iley, B.
J., Grose-Hodge, M., Tyler, S. L., Kapp, S. K., Gourdon-Kanhukamwe, A.,
Grafton-Clarke, D., Yeung, S. K., Shaw, J. J., Hartmann, H., \&
Dokovova, M. (2022). Bridging Neurodiversity and Open Scholarship: How
Shared Values Can Guide Best Practices for Research Integrity, Social
Justice, and Principled Education. \emph{MetaArXiv}.
\url{https://doi.org/10.31222/osf.io/k7a9p}

Fille, R. (2022). \emph{Rainbowashing: Does it Impact Purchase
Intention?} {[}Honors Thesis, Bryant University{]}.
\url{https://digitalcommons.bryant.edu/honors_marketing/46}

Frick, C., \& Honold, C. (2022). Gendersensible Sprache im Kontakt mit
Bibliotheksnutzenden. \emph{027.7 Zeitschrift Für Bibliothekskultur /
Journal for Library Culture}, 9(4).
\url{https://doi.org/10.21428/1bfadeb6.58e38319}

Gerlach, S. (2023). \emph{How queer is the library (not)? - Die
bibliothekswissenschaftliche Rezeption von LGBTIQ*: ein Vergleich
zwischen Deutschland und den USA.} {[}Master's thesis,
Cologne University of Applied Sciences{]}.
\url{https://nbn-resolving.org/urn:nbn:de:hbz:79pbc-opus-20933}

Glass, V. Q. (2022). Queering Relationships: Exploring Phenomena of
Asexually Identified Persons in Relationships. \emph{Contemporary Family
Therapy}, 44(4), 344-359.
\url{https://doi.org/10.1007/s10591-022-09650-9}

Government Equalities Office (2018). \emph{National LGBT Survey:
Research Report.} Government Equalities Office.
\url{https://assets.publishing.service.gov.uk/media/5b3b2d1eed915d33e245fbe3/LGBT-survey-research-report.pdf}

Heinze, J. L. (2021, January 24). Fact Sheet on Injustice in the LGBTQ
community. National Sexual Violence Resource Center.
\url{https://www.nsvrc.org/blogs/fact-sheet-injustice-lgbtq-community}

Heizereder, S. (2023). Der Twitter-Mob wütet. \emph{BuB - Forum
Bibliothek und Information}, 75(6), 257.

Heudorfer, K., Steinbauer, M. M., Schröter, A. M., \& Brack, G. (2023,
June 13). \emph{Drag-Lesung für Kinder: 500 Menschen für ein buntes
München}. BR24. Retrieved February 20, 2024, from
\url{https://www.br.de/nachrichten/bayern/drag-lesung-fuer-kinder-500-menschen-fuer-ein-buntes-muenchen,Th3oG7z}

Hutchinson, J., \& Nakatomi, J. (2023). Improving Subject Description of
an LGBTQ+ Collection. \emph{Cataloging \& Classification
Quarterly},\emph{61}(3-4), 380-394 .
\url{https://doi.org/10.1080/01639374.2023.2229828}

Ipsos. (2023, June 1). \emph{LGBT+ Pride 2023}. Retrieved February 20,
2024, from
\url{https://www.ipsos.com/en/pride-month-2023-9-of-adults-identify-as-lgbt}

Keite, U. (2024). Gendersensible und diskriminierungsfreie Sprache. In
Berufsverband Information Bibliothek / Kommission für One-Person
Libraries, \emph{OPL-Checklisten}, Checkliste 47. Retrieved February 20,
2024, from
\url{https://www.bib-info.de/fileadmin/public/Dokumente_und_Bilder/Komm_OPL/Checklisten/check47.pdf}

Kennon, P. (2021). Asexuality and the Potential of Young Adult
Literature for Disrupting Allonormativity. \emph{The International
Journal of Young Adult Literature}, 2(1), 1-24.
\url{https://doi.org/10.24877/IJYAL.41}

Kitzie, V., Vera, A.N., Lookingbill, V. and Wagner, T.L. (2024), \enquote{What
is a wave but 1000 drops working together?}: The role of public
libraries in addressing LGBTQIA+ health information
disparities\textquotesingle, \emph{Journal of Documentation}, 80(2),
533-551. \url{https://doi.org/10.1108/JD-06-2023-0122}

Laudenbach, P. (2023, August 30). Rechte Angriffe auf Bibliotheken:
Bücher mit Messern zerschnitten - Kulturkampf. \emph{Süddeutsche
Zeitung.} Retrieved February 20, 2024, from
\url{https://www.sueddeutsche.de/kultur/bibliotheken-rechte-angriffe-antisemitismus-buecher-zerschneiden-1.6177557}

Leyrer, K. (2014). Das Geschlecht spukt in der Stadtbibliothek: Ein
Aufruf für genderneutrale Bibliotheksangebote, \emph{LIBREAS Library
Ideas,} 25, 76-79. \url{https://doi.org/10.25595/401}

Lieb, S. (2023). \emph{Alle(s) Gender: Wie kommt das Geschlecht in den
Kopf?} (1st edition). Querverlag.

Log, A. (2022, November 15). Answering questions on the label \enquote{queer}.
\emph{The Clock}.
\url{https://www.plymouth.edu/theclock/answering-questions-on-the-label-queer/}

Mefebue, A. B. (2016). Umgang mit sozialer Diversität in der
Bibliotheksarbeit -- eine empirische Untersuchung. In K. Futterlieb \&
J. Probstmeyer (Hrsg.), \emph{Diversity Management und interkulturelle
Arbeit in Bibliotheken} (pp.~43-74). De Gruyter.
\url{https://doi.org/10.1515/9783110338980-005}

Mehra, B. (2011). Integrating LGBTIQ Representation Across the Library
and Information Science Curriculum: A Strategic Framework for
Student-Centered Interventions. In E. Greenblatt (Ed.), \emph{Serving
LGBTIQ library and archives users: Essays on outreach, service,
collections and access} (pp.~298-309). McFarland \& Company, Inc,
Publishers.

Miebach, E. (2023, June 13). Eine Kinderbuchlesung im Wahlkampfstrudel.
\emph{ZDFheute}. Retrieved February 20, 2024, from
\url{https://www.zdf.de/nachrichten/panorama/muenchen-dragqueen-lesung-diskussion-demo-100.html}

Minkov, M. (2021). Eine Pause von der Welt. \emph{Zeitgeister: Das
Kulturmagazin des Goethe-Instituts.} Retrieved February 20, 2024, from
\url{https://www.goethe.de/prj/zei/de/art/22554555.html}

Mobile Beratung gegen Rechtsextremismus Berlin. (2023). \emph{Alles nur
leere Worte? - Zum Umgang mit dem Kulturkampf von rechts in
Bibliotheken.} Retrieved February 20, 2024, from
\url{https://www.mbr-duesseldorf.de/fileadmin/content/medien/230715_MBR_Broschuere_Bibliotheken_online.pdf}

Nault, C. (2023). On Fear, Professionalism, and Being That Trans Guy
Library Technician. In K. K. Adolpho, S. G. Krueger, \& K. McCracken
(Eds.), \emph{Trans and gender diverse voices in libraries} (pp.~45-51).
Library Juice Press.

Orsborn, C. E. (2022). \emph{A Golden Age of Censorship: LGBTQ Young
Adult Literature in High School Libraries} {[}Electronic thesis or
dissertation, Ohio Dominican University{]}. OhioLINK Electronic Theses
and Dissertations Center.
\url{http://rave.ohiolink.edu/etdc/view?acc_num=oduhonors1669994581915957}

Riggle, E. D. B., Rostosky, S. S., Black, W. W., \& Rosenkrantz, D. E.
(2017). Outness, concealment, and authenticity: Associations with LGB
individuals\textquotesingle{} psychological distress and well-being.
\emph{Psychology of Sexual Orientation and Gender Diversity}, 4(1),
54-62. \url{https://doi.org/10.1037/sgd0000202}

Roberto, K. R. (2011). Passing Tips and Pronoun Police: A Guide to
Transitioning at Your Local Library. In T. M. Nectoux (Ed.), \emph{Out
behind the desk: Workplace issues for LGBTQ librarians} (pp.~121-127).
Library Juice Press.

Siggelkow, C., \& Reveland, P. (2023, July 17).
\emph{Queerfeindlichkeit: Verstärkte Mobilisierung gegen queere
Menschen}. Tagesschau. Retrieved February 20, 2024, from
\url{https://www.tagesschau.de/faktenfinder/kontext/queerfeindlichkeit-desinformation-100.html}

Sinclair-Palm, J., \& Chokly, K. (2023).
'It's a giant faux pas\textquotesingle:
Exploring young trans people's beliefs about deadnaming
and the term deadname. \emph{Journal of LGBT Youth}, 20(2), 370-389.
\url{https://doi.org/10.1080/19361653.2022.2076182}

Sparber, S. (2016). What's the frequency, Kenneth? --
Eine (queer)feministische Kritik an Sexismen und Rassismen im
Schlagwortkatalog. \emph{Mitteilungen der Vereinigung Österreichischer
Bibliothekarinnen und Bibliothekare}, 69(2), 236-243.
\url{https://doi.org/10.31263/voebm.v69i2.1629}

Stringer-Stanback, K., \& Jackson, L. (2023). Remixing LIS Leadership:
Considering Gender-Vari\-ant BIPOC - Are we there yet? In K. K. Adolpho,
S. G. Krueger, \& K. McCracken (Eds.), \emph{Trans and gender diverse
voices in libraries} (pp.~455-465). Library Juice Press.

The Roestone Collective (2014). Safe Space: Towards a
Reconceptualization. \emph{Antipode}, 46(5), 1346-1365.
\url{https://doi.org/10.1111/anti.12089}

Thorne, N., Yip, A. K.-T., Bouman, W. P., Marshall, E., \& Arcelus, J.
(2020). The terminology of identities between, outside and beyond the
gender binary - A systematic review. In J. Motmans, T. O. Nieder, \& W.
P. Bouman (Eds.), \emph{Non-binary and Genderqueer Genders} (pp.~20-36).
Routledge.

Timmo D. (2022). \emph{Wheel of Privilege and Power}. Center for
Teaching, Learning \& Mentoring.
\url{https://kb.wisc.edu/instructional-resources/page.php?id=119380ional-resources/page.php?id=119380}

Vries, L. D., Fischer, M., Kasprowski, D., Kroh, M., Kühne, S., Richter,
D., \& Zindel, Z. (2020). LGBTQI*-Menschen am Arbeitsmarkt: Hoch
gebildet und oftmals diskriminiert. \emph{DIW Wochenbericht}, 36,
620--627. \url{https://doi.org/10.18723/DIW_WB:2020-36-1}

Walters, J. (2023). Standing Out. In K. K. Adolpho, S. G. Krueger, \& K.
McCracken (Eds.), \emph{Trans and gender diverse voices in libraries}
(pp. 175-180). Library Juice Press.

Werner, K. U. (2024). Gut gemeint ist nicht immer gut. \emph{BuB - Forum
Bibliothek und Information}, 76(1), 21.

Wexelbaum, R. (2019). The Reading Habits and Preferences of LGBTIQ+
Youth. \emph{The International Journal of Information, Diversity, \&
Inclusion (IJIDI)}, 3(1),
112-129.\url{https://doi.org/10.33137/ijidi.v3i1.32270}

Wilk, A., \& Vincent, J. (2018, April 3). Respecting anonymity through
collection development. \emph{Open Shelf}. Retrieved February 20, 2024,
from
\url{https://open-shelf.ca/180403-respecting-anonymity-through-collection-development/}

Wright, D. (2024, February 23). \emph{Heartstopper! Sustaining the
Library-LGBTQ Love Affair.} Making collections accessible and diverse:
current approaches to audience engagement {[}Video{]}. YouTube.
\url{https://youtu.be/6t4yJGK4QGY?t=2282}

Zeuner, P., Buchert, C., Fischer, Y., Baumann, N., Frick, C., \&
Ramünke, S. (2024). Bibliotheken als Safe(r) Spaces für die LGBTQIA+
Community? Hands-on Lab auf der BiblioCon 2024. API Magazin, 5(2),
Article 2. \url{https://doi.org/10.15460/apimagazin.2024.5.2.209}

%autor
\begin{center}\rule{0.5\linewidth}{0.5pt}\end{center}

\textbf{Autor*innen}

Claudia Frick, Technische Hochschule Köln, Köln, Deutschland, claudia.frick@th-koeln.de\newline
\url{https://orcid.org/0000-0002-5291-4301}

Philipp Zeuner, Bundesinstitut für Berufsbildung, Bonn, Deutschland\newline
\url{https://orcid.org/0000-0002-1307-1145}

Caleb Buchert, Technische Hochschule Köln, Köln, Deutschland\newline
\url{https://orcid.org/0009-0004-0470-1311}

Daniela Markus, Universitätsbibliothek Kiel, Christian-Albrechts-Universität zu Kiel, Kiel,\linebreak Deutschland, \url{https://orcid.org/0009-0008-3514-9450}

Norma Fötsch, University Library, Radboud University, Nijmegen, The Netherlands\newline
\url{https://orcid.org/0009-0009-7009-5520}

Yvonne Fischer, Stadtbibliothek Köln, Köln, Deutschland\newline
\url{https://orcid.org/0009-0004-6326-8191}

Emma Wieseler, Technische Hochschule Köln, Köln, Deutschland\newline
\url{https://orcid.org/0009-0001-0738-9747}

Sabrina Ramünke, Universitätsbibliothek Freie Universität Berlin, Berlin, Deutschland\newline
\url{https://orcid.org/0000-0003-4091-7588}

Nik Baumann, Stadt- und Schulbücherei Gunzenhausen und Landesfachstelle für das öffentliche Bibliothekswesen Bayern - Nürnberg, Deutschland, \url{https://orcid.org/0009-0006-1220-079X}

\end{document}