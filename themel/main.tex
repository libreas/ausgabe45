\documentclass[a4paper,
fontsize=11pt,
%headings=small,
oneside,
numbers=noperiodatend,
parskip=half-,
bibliography=totoc,
final
]{scrartcl}

\usepackage[babel]{csquotes}
\usepackage{synttree}
\usepackage{graphicx}
\setkeys{Gin}{width=.4\textwidth} %default pics size

\graphicspath{{./plots/}}
\usepackage[ngerman]{babel}
\usepackage[T1]{fontenc}
%\usepackage{amsmath}
\usepackage[utf8x]{inputenc}
\usepackage [hyphens]{url}
\usepackage{booktabs} 
\usepackage[left=2.4cm,right=2.4cm,top=2.3cm,bottom=2cm,includeheadfoot]{geometry}
\usepackage[labelformat=empty]{caption} % option 'labelformat=empty]' to surpress adding "Abbildung 1:" or "Figure 1" before each caption / use parameter '\captionsetup{labelformat=empty}' instead to change this for just one caption
\usepackage{eurosym}
\usepackage{multirow}
\usepackage[ngerman]{varioref}
\setcapindent{1em}
\renewcommand{\labelitemi}{--}
\usepackage{paralist}
\usepackage{pdfpages}
\usepackage{lscape}
\usepackage{float}
\usepackage{acronym}
\usepackage{eurosym}
\usepackage{longtable,lscape}
\usepackage{mathpazo}
\usepackage[normalem]{ulem} %emphasize weiterhin kursiv
\usepackage[flushmargin,ragged]{footmisc} % left align footnote
\usepackage{ccicons} 
\setcapindent{0pt} % no indentation in captions
\usepackage{xurl} % Breaks URLs

%%%% fancy LIBREAS URL color 
\usepackage{xcolor}
\definecolor{libreas}{RGB}{112,0,0}

\usepackage{listings}

\urlstyle{same}  % don't use monospace font for urls

\usepackage[fleqn]{amsmath}

%adjust fontsize for part

\usepackage{sectsty}
\partfont{\large}

%Das BibTeX-Zeichen mit \BibTeX setzen:
\def\symbol#1{\char #1\relax}
\def\bsl{{\tt\symbol{'134}}}
\def\BibTeX{{\rm B\kern-.05em{\sc i\kern-.025em b}\kern-.08em
    T\kern-.1667em\lower.7ex\hbox{E}\kern-.125emX}}

\usepackage{fancyhdr}
\fancyhf{}
\pagestyle{fancyplain}
\fancyhead[R]{\thepage}

% make sure bookmarks are created eventough sections are not numbered!
% uncommend if sections are numbered (bookmarks created by default)
\makeatletter
\renewcommand\@seccntformat[1]{}
\makeatother

% typo setup
\clubpenalty = 10000
\widowpenalty = 10000
\displaywidowpenalty = 10000

\usepackage{hyperxmp}
\usepackage[colorlinks, linkcolor=black,citecolor=black, urlcolor=libreas,
breaklinks= true,bookmarks=true,bookmarksopen=true]{hyperref}
\usepackage{breakurl}

%meta
%meta

\fancyhead[L]{S. Themel\\ %author
LIBREAS. Library Ideas, 45 (2024). % journal, issue, volume.
\href{https://doi.org/10.18452/...}{\color{black}https://doi.org/10.18452/...}
{}} % doi 
\fancyhead[R]{\thepage} %page number
\fancyfoot[L] {\ccLogo \ccAttribution\ \href{https://creativecommons.org/licenses/by/4.0/}{\color{black}Creative Commons BY 4.0}}  %licence
\fancyfoot[R] {ISSN: 1860-7950}

\title{\LARGE{The Sound of Libraries -- ein Hörbild der Stadt Wien Büchereien}}% title
\author{Sarah Themel} % author

\setcounter{page}{1}

\hypersetup{%
      pdftitle={The Sound of Libraries –- ein Hörbild der Stadt Wien Büchereien},
     pdfauthor={Sarah Themel},
      pdfcopyright={CC BY 4.0 International},
      pdfsubject={LIBREAS. Library Ideas, 45 (2024).},
      pdfkeywords={Wien, Büchereien, Pandemie, Covid-19},
      pdflicenseurl={https://creativecommons.org/licenses/by/4.0/},
      pdfurl={https://doi.org/10.18452/...},
      pdfdoi={10.18452/...},
      pdflang={de},
      pdfmetalang={de}
     }



\date{}
\begin{document}

\maketitle
\thispagestyle{fancyplain} 

%abstracts

%body
Das Hörbild \enquote{The Sound of Libraries -- ein Hörbild der Stadt
Wien - Büchereien} entstand im Rahmen meiner Abschlussarbeit im
\enquote{Lehrgang für hauptamtliche Bibliothekar*innen} durch den BVÖ
(Büchereiverband Österreich). Die Arbeit trägt den Titel \emph{\enquote{Wir
hören einander zu} - Lehrlingsprojekt \enquote{Hörbild}. Partizipation
als Notwendigkeit hin zur Demokratisierung von Arbeitsabläufen} (Themel,
2023).\footnote{Das Hörbild ist unter \url{https://www.youtube.com/watch?v=z-MIt4zpqjc} verfügbar.}

Mit März 2020 und der Coronapandemie änderte sich mein ursprüngliches
Vorhaben, ein Hörspiel mit einigen Lehrlingen der Stadt Wien -
Büchereien zu erarbeiten, damit Schreibprozesse sowie audiomediale
Arbeit zu begleiten und zu beobachten, mit dem Ergebnis einer tonalen
Erzählung. Alltägliche Abläufe und die diesen immanenten
Prozesshaftigkeiten -- Arbeit, Kinderbetreuung, soziales Miteinander et
cetera -- verloren durch die Pandemie ihre beschworene Manifestation.

Ein großes Unglück? Oder unsere größte Chance? Ich entschloss mich, am
Audioprojekt festzuhalten, inhaltlich jedoch ein Stimmungsbild der
pandemischen Zeit darzustellen. Anhand eines mit den Lehrlingen
gemeinsam ausgearbeiteten Fragebogens wurden mit 18 Mitarbeiter*innen
der Stadt Wien - Büchereien Interviews zum Thema \enquote{Corona}
durchgeführt.

Aus der anfänglichen Idee eines \enquote{Klangnarrativs} entwickelte
sich die Absicht, ein \enquote{Hörbild} entstehen zu lassen. Anhand der
Interviews sollte rein akustisch-auditiv zur zunächst grob umrissenen
Thematik \enquote{Die Stadt Wien - Büchereien in Zeiten von Corona}
berichtet werden.

Im Zentrum des Projekts, auch schon des ursprünglichen, sollte die
Ermöglichung von aktiver Teilhabe der Lehrlinge an einem künstlerischen,
in die bibliothekarische Arbeit eingebetteten Prozesses stehen. Dadurch
eröffnen sich für die Auszubildenden einerseits neue Blickwinkel im
Umgang mit Kunst und Kultur, welche sie in ihre bibliothekarische Arbeit
(zum Beispiel der Vermittlungsarbeit) mit einfließen lassen können.
Andererseits bedeutet eine solche Teilhabe für diese Gruppe auch eine
wichtige Voraussetzung für weitere, selbstbewusste Partizipation und
Mitgestaltung von gesellschaftlichen und damit auch arbeitsrelevanten
Prozessen.

In der gemeinsamen Arbeit mit den Lehrlingen (ein Schreibworkshop samt
Vortrag eines Sounddesigners, das Erarbeiten der Fragen für die
Interviews, der Durchführung einiger davon) und in Bezug auf das
inhaltliche Anliegen der Aufnahmen, stellten sich mir einige weitere
Fragen, zum Beispiel die nach der Bedeutung von Partizipation am
Arbeitsplatz, demokratischen Arbeitsabläufen, kreativer
Gemeinschaftlichkeit und der Notwendigkeit darum. Bibliotheken sind Orte
der Information, Orte des Austauschs von Wissen -- und in jedem Falle
sind sie Orte der Begegnung. Wie wollen wir einander begegnen? Wäre die
Bibliothek als Arbeitsplatz nicht ein wunderbarer Ort, um partizipative,
demokratische, kreative Begegnungen zu erproben? Besonders Auszubildende
sollten dabei \enquote{aus den Vollen} schöpfen können.

Öffentliche Büchereien verstehen sich schon längst nicht mehr nur als
Orte des reinen Wissens- sowie Informationstransfers. \enquote{Der
dritte Ort}, Barrierefreiheit, Diversität -- das sind nur ein paar
Schlagworte, mit welchen sich die Bücherei neu identifiziert. Von der
Leseförderung über den Maker Space hin zur gesellschaftlichen Teilhabe
wird versucht, durch das eigene Angebot gesellschaftlich relevante
Brücken zu bauen. Das ist nicht nur zeitgemäß, sondern eine wichtige und
unbedingt zu erfüllende Aufgabe. Kreativ, mit viel Know-how
ausgestattet, allgemeinen Bedürfnissen nachspürend,
gesellschaftspolitische Trends erkennend, soll so der unter anderem auch
kommunalen Funktion der öffentlichen Bücherei nachgekommen werden. Orte
des Wohlfühlens, des Experiments, der kreativen Entfaltung, der
Barrierefreiheit, ein Raum ohne Konsumzwang -- werden immer mehr an
Bedeutung gewinnen -- so auch die öffentliche Bücherei.

\hypertarget{warum-das-thema-corona-von-der-notwendigkeit-einer-gemeinsamen-aufarbeitung}{%
\section{\texorpdfstring{Warum das Thema \enquote{Corona}?: Von
der Notwendigkeit einer gemeinsamen
Aufarbeitung}{Warum das Thema ``Corona''?: Von der Notwendigkeit einer gemeinsamen Aufarbeitung}}\label{warum-das-thema-corona-von-der-notwendigkeit-einer-gemeinsamen-aufarbeitung}}

Beinahe alles, was wir vor der Pandemie als Routine betrachteten, was
unseren Alltag strukturierte und unser gewohntes Leben aufrechterhielt,
war von einem Tag auf den anderen nicht mehr greifbar. Fast drei Jahre
lang sahen wir uns in permanenter Alarmbereitschaft und damit im
Ausnahmezustand. Später kamen noch dazu: ein Krieg mitten in Europa,
Energieengpässe, Inflation et cetera, ein großes Fragezeichen, was
unsere gemeinsame Zukunft betrifft.

Auch die Mitarbeiter*innen der Stadt Wien - Büchereien sahen sich mit
einer Aushebelung aller davor geltenden Praktiken konfrontiert. Alle
Standorte (38 Zweigstellen) wurden geschlossen; der eigentlichen
Funktion, dem Medienverleih, konnte nicht mehr nachgekommen werden.
Relativ rasch wurde die Möglichkeit geschaffen, sogar kostenlos
Literatur für (fast) alle über die virtuelle Bücherei zugänglich zu
machen.

\begin{quote}
\enquote{Im ersten Lockdown im März 2020 hatten die Stadt Wien --
Büchereien die Aufgabe, allen Wiener*innern {[}sic!{]} ihre vielfältigen
Angebote weiter zugänglich zu machen. Da ein Besuch der Standorte nicht
möglich war, wurden diese in den virtuellen Raum verlagert.} (Schneider
\& Volf, 2021, S. 473)
\end{quote}

Dieses Angebot wurde auch zahlreich in Anspruch genommen:

\begin{quote}
\enquote{Mehr als 15.000 Menschen waren binnen weniger Tage
eingeschrieben und konnten sich im Lockdown an den virtuellen Angeboten
bedienen. Die Ausleihzahlen im April 2020 verdoppelten sich auf über
100.000, PressReader und Austria Kiosk erlebten die höchsten monatlichen
Zugriffe bisher.} (Schneider \& Volf, 2021, S. 474)
\end{quote}

Die Tatkraft vieler Mitarbeiter*innen konnte dies trotz der Umstände
ermöglichen. Auch gelang es, Veranstaltungen für alle Altersgruppen in
den virtuellen Raum zu verlegen und zu etablieren, um so eine
Nutzer*innenbindung aufzubauen beziehungsweise bestehen zu lassen.

\begin{quote}
\enquote{Neue Nutzer*innen konnten durch die systematische Bespielung
weiterer vorhandener (YouTube, Podcasts) sowie neuer Kanäle (Instagram)
gewonnen werden. Rasch riefen die Büchereien eine Schiene von
Online-Lesungen unter dem Titel ‚Corona-Lesungen‛ ins Leben. Das Format
bestand aus einer Mischung von Lesung und Interview mit österreichischen
Autor*innen. Um ein durchgängiges Programm (bis zu 3 Lesungen/Woche) zur
Verfügung stellen zu können, wurde auf Initiative der Büchereien eine
Kooperation mit Alter Schmiede, Österreichischer Gesellschaft für
Literatur und dem Hauptverband des Buchhandels eingegangen. {[}\ldots{]}
den Zuspruch zu ihren bestehenden Online-Kanälen deutlich ausbauen und
neue Kanäle etablieren. Gesamt verfügen die Social-Media-Kanäle der
Büchereien mittlerweile über rund 85.000 Abonnent*innen. Die Reichweite geht in die mehreren 100.000.} (Schneider \&Volf, 2021, S. 475)
\end{quote}

Soweit eine Erfolgsgeschichte in krisenhaften Zeiten, welche sogar mit
dem \enquote{Goldenen Staffelholz} der Stadt Wien ausgezeichnet wurde.
Mit den schrittweisen Öffnungen (dazwischen immer wieder Lockdowns)
musste immer wieder flexibel und angepasst auf die jeweiligen
Bedürfnisse reagiert werden. Mit Click \& Collect konnte einer
teilweisen Öffnung begegnet werden: In beinahe allen Zweigstellen
konnten Medien (bis zu 20 Stück) kostenlos vorbestellt und unter
Einhaltung der jeweiligen Maßnahmen (zum Beispiel FFP2-Maske) abgeholt
werden. Vom Waschen der Medien, Personenbeschränkungen und der
Überprüfung dieser, vom verkürzten Aufenthalt der Leser*innen in den
Büchereien, vom Gesichtsschild, der Maske und den Handschuhen -- dem
temporären Verlust beinahe jeglicher kollegialen Begegnung -- all dem
wurde mit viel Flexibilität und Durchhaltevermögen durch die
Mitarbeiter*innen der Stadt Wien - Büchereien erfolgreich begegnet. Nach
außen hin hat also alles geklappt, das Angebot konnte weitestgehend
erfolgreich transportiert werden und fand damit auch entsprechenden
Anklang.

Wie aber wirkte und wirkt diese pandemische Krise mit all ihren Effekten
nach innen hinein? Wie ging und geht es den Mitarbeiter*innen in dieser
unsicheren Zeit? Was bedeutete diese Zeit für Mitarbeiter*innen mit
Kindern, für Alleinstehende, für Mitarbeiter*innen mit Vorerkrankungen
oder kranken Angehörigen, für sehr junge Mitarbeiter*innen (Lehrlinge)?
Wie erlebten sie diese Zeiten ununterbrochener Unsicherheit gepaart mit
ständig geforderter Flexibilität? Hat sich ihre Vorstellung von Arbeit
und ihre Einstellung dazu geändert? Sehen sie darin möglicherweise auch
einen Gewinn, eine Chance für eine positive Veränderung? Diese Fragen,
und viele mehr, stellte und stelle ich mir permanent selbst und suchte
einen Austausch. Mit dem für mich anstehenden Projekt ergab sich somit
die Möglichkeit, einen solchen Dialog herstellen zu können. Um weiteren
krisenhaften Zeiten vorzubeugen, müssen bestehende und vergangene auch
ihre Aufarbeitung finden.

\begin{quote}
\enquote{Nichtsdestotrotz ist eine historische Wahrheit, dass die großen
strukturellen Transformationen der Gesellschaft in der Geschichte häufig
das Resultat tiefer sozialer Krisenkonstellationen wie Wirtschaftskrise
oder Krieg waren, weil diese ein Business-as-usual verunmöglichten.
Krisen sind also tatsächlich gefährliche Wendepunkte, aber insofern
diese geschichtsoffen sind, bieten sie eben auch Chancen für eine
Entwicklung hin zum Besseren.} (Solty, 2021, S. 672)
\end{quote}

Im gemeinsam gestalteten Hörbild sollten die Stimmen der
Mitarbeiter*innen ihr Gehör finden und diese Zeit für die Stadt Wien -
Büchereien damit auch ein Stück weit ihre Dokumentation finden.

\hypertarget{literatur}{%
\section{Literatur}\label{literatur}}

Schneider, Magdalena Martha Maria und Volf, Patrik-Paul (2021)
\enquote{Wir gehen viral! Die Stadt Wien -- Büchereien im ersten
Lockdown}, Mitteilungen der Vereinigung Österreichischer
Bibliothekarinnen und Bibliothekare, 73(3-4), S. 473--478.
\url{https://doi.org/10.31263/voebm.v73i3-4.5355}.

Solty, Ingar (2021) Krise als Krise / Krise als Chance. Wie aus dem
Elend der Gegenwart eine neue demokratischere, sozialere und
ökologischere Produktions- und Lebensweise entstehen könnte. In: D.F.
Bertz {[}Hrsg.{]}: Die Welt nach Corona. Von den Risiken des
Kapitalismus, den Nebenwirkungen des Ausnahmezustands und der kommenden
Gesellschaft. Berlin: Bertz + Fischer Gbr.

Themel, Sarah (2023) \enquote{Wir hören einander zu} - Lehrlingsprojekt
\enquote{Hörbild}. Partizipation als Notwendigkeit hin zur
Demokratisierung von Arbeitsabläufen. Projektarbeit im Rahmen der
hauptamtlichen Ausbildung für Bibliothekar*innen, 3. Lehrgang,
2019--2021. URL: \url{https://projektarbeiten.bvoe.at/ThemelSarah.pdf}

%autor
\begin{center}\rule{0.5\linewidth}{0.5pt}\end{center}

\textbf{Sarah Themel} ist Germanistin und arbeitet bei den Stadt Wien
Büchereien als Kinderbibliothekarin. Sie schreibt und spielt
``feministisches Kasperltheater''.

\end{document}