\textbf{Kurzfassung:} Der Beitrag stellt in aller Kürze das Konzept der
Soundscape vor, wie es vor allem von R. Murray Schafer (1977) entwickelt
worden ist. Ausgehend von Schafer wird in der Folge argumentiert, dass
es sich bei der Bibliothek um eine spezifische Ausprägung einer
Soundscape handelt, die im Kontrast zur gesellschaftlichen Umwelt als
klangliche Heterotopie (Hi-Fi-Soundscape) beschrieben werden kann. Diese
Hi-Fi-Umgebung trägt maßgeblich zur Entstehung einer lernförderlichen
Atmosphäre in Bibliotheken bei, die unter anderem mit Begriffen wie
\enquote{Bibliothekskonzentration} oder \enquote{Out-of-the-Box-Konzentration}
(Fansa, 2008) beschrieben worden ist und als Alleinstellungsmerkmal der
Einrichtung gilt. Das Soundscape-Konzept ermöglicht hier eine neue
analytische Perspektive auf diese Zusammenhänge, wodurch ein Mehrwert
unter anderem für die bibliothekarischen Diskussionen um den physischen
Lernort Bibliothek generiert werden kann. Zum Abschluss werden einige
potentielle Anschlussthemen skizziert.

\begin{center}\rule{0.5\linewidth}{0.5pt}\end{center}

\textbf{Abstract:} This article briefly introduces the concept of the
soundscape as primarily conceptualized by R. Murray Schafer (1977).
Based on Schafer, it is argued that the library is a specific form of
soundscape that can be described as a sonic heterotopia
(Hi-Fi-Soundscape) in contrast to the social environment. This hi-fi
environment contributes significantly to an atmosphere conducive to
learning, which has been described with terms such as \enquote{library
concentration} (Bibliothekskonzentration) or \enquote{out-of-the-box
concentration} (Fansa, 2008) and is considered a unique selling point
of libraries. The soundscape concept enables a new analytical
perspective on these contexts, which can add value to discussions about
the physical learning location of the library as a whole. Finally, some
potential follow-up topics are outlined.